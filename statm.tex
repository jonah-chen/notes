\documentclass{article}
\usepackage[a4paper,margin=0.5in]{geometry}
\usepackage{amsmath}
\usepackage{amssymb}
\usepackage{physics}
\usepackage{physoly}
\usepackage{verbatim}
\usepackage{siunitx}

\newcommand{\bigspace}{\:\:\:\:\:\:\:\:\:\:\:\:\:\:\:\:\:\:\:\:\:\:\:\:\:\:\:\:\:\:}
\newcommand{\smallspace}{\:\:\:\:\:\:\:\:\:\:}
\newcommand{\ve}{\mathbf}

\title{Notes on Statistical Mechanics by Pathria}
\author{Jonah Chen}

\begin{document}
\maketitle
\tableofcontents
Natural units will be used hahahaha $\hbar=c=k_B=1$
\section{Axioms and Derivation of Thermodynamic Quantities and Formule}
\begin{itemize}
    \item The axiom of statistial mechanics is called "equal \textit{a priori} probabilities", that states:
    \begin{verbatim}
        When there are no additional constraints, a given macrostate of the system at any
        time is equally likely to be found in any one of its microstates.
    \end{verbatim}

    \item The number of macrostates is written as a function of the extensive variables $\Omega(N,V,E)$ representing number of particles, volume and total energy respectively. 

    \begin{definition}
        Equilibrium is when a system is at a macrostate with the maximum number of microstates. Which is equivilent to when $\Omega$ is maximized.
    \end{definition}

    \item Firstly we will derive temperature and entropy from our axiom and definition of equilibrium.
    \begin{derivation}
        Take two physical systems at equilibrium: $A_1$ at $N_1, V_1, E_1$  and $A_2$ at $N_2, V_2, E_2$. Now have them contact such that the the volume doesn't change but energy is allowed to be transfered from one system to another.

        Energy is conserved, thus $E^0$ is constant,
        \begin{equation}
            E^0\equiv E_1+E_2 \label{1}
        \end{equation}

        When the two systems reach equilibrium again at equilibrium energies $\overline{E_1}$ and $\overline{E_2}$ respectively, by definition the total number of macrostates $\Omega_1(E_1)\Omega_2(E_2)$ is maximized. Thus,
        \begin{equation}
            \frac{\partial}{\partial E_1}(\Omega_1(E_1)\Omega_2(E_2))=0
        \end{equation}
        Using the product rule and chain rule,
        \begin{equation}
            \partialderivative{\Omega_1(E_1)}{E_1}\Bigg|_{E_1=\overline{E_1}}\Omega_2(E_2)+\Omega_1(E_1)\partialderivative{\Omega_2(E_2)}{E_2}\Bigg|_{E_2=\overline{E_2}}\derivative{E_2}{E_1}=0\label{equilibrium-conducting}
        \end{equation}
        Noting that $dE_2/dE_1=-1$ from equation \eqref{1}
        \begin{equation}
            \partialderivative{\Omega_1(E_1)}{E_1}\Bigg|_{E_1=\overline{E_1}}\Omega_2(E_2)=\Omega_1(E_1)\partialderivative{\Omega_2(E_2)}{E_2}\Bigg|_{E_2=\overline{E_2}}
        \end{equation}
        Taking advantage of the fact that the derivative of $\log u$ is $u'/u$
        \begin{equation}
            \partialderivative{\log(\Omega_1(E_1))}{E_1}\Bigg|_{E_1=\overline{E_1}}=\partialderivative{\log(\Omega_2(E_2))}{E_2}\Bigg|_{E_2=\overline{E_2}}\label{equilibrium-conducting-2}
        \end{equation}
        Notice that these it is only possible to be in equilibrium when these two quantities are equal. In general, define
        \begin{equation}
            \beta\equiv\left(\partialderivative{\log\Omega}{E}\right)_{N,V,E=\overline{E}}
        \end{equation} 
        This resembles the behavior of the thermodynamic temperature and should be somewhat related. Furthermore, recall from the second law
        \begin{equation}
            \left(\partialderivative{S}{E}\right)_{N,V}=\frac{1}{T}
        \end{equation}
        And thus,
        \begin{equation}
            \frac{\Delta S}{\Delta \log\Omega}=\frac{1}{\beta T}=\text{constant}
        \end{equation}
        This constant is known as Boltzmann's constant $k$, and thus
        \begin{equation}
            S=k\log\Omega\label{entropy}
        \end{equation}
        This is the definition of entropy. Zero entropy represents when there is only one microstate, which is consistent.   
    \end{derivation}

    \item Now the other intensive quantities pressure and chemical potential can be realized with a similar approach.

    \begin{derivation}
        Recall the basic formula of thermodynamics
        \begin{equation}
            \dd E=T\dd S-P\dd V+\mu\dd N \label{basic:thermo}
        \end{equation}
        Now assume that the barrier between the two systems is movable. Then, the volumes are able to change with $V_1+V_2$ constant. From the similar derivation above
        \begin{equation}
            \partialderivative{\log\Omega_1}{V_1}\Bigg|_{V_1=\overline{V_1}}=\partialderivative{\log\Omega_2}{V_2}\Bigg|_{V_2=\overline{V_2}}
        \end{equation}
        In a similar fashion, define
        \begin{equation}
            \eta\equiv\left(\partialderivative{\log\Omega}{V}\right)_{N,E,V=\overline{V}}
        \end{equation}
        This quantity can be rewritten as 
        \begin{equation}
            \eta=\partialderivative{\log\Omega}{E}\derivative{E}{V}=\frac{P}{kT}\label{13}
        \end{equation}
        as $\partialderivative{\log\Omega}{E}$ is $\frac{1}{kT}$ from above and $dE/dV$ is the thermodynamic pressure $P$ from \eqref{basic:thermo}.

        Similarly, if particles are allowed to be transfered from one system to the other then at equilibrium,
        \begin{equation}
            \partialderivative{\log\Omega_1}{N_1}\Bigg|_{N_1=\overline{N_1}}=\partialderivative{\log\Omega_2}{N_2}\Bigg|_{N_2=\overline{N_2}}
        \end{equation}
        Define
        \begin{equation}
            \zeta\equiv\left(\partialderivative{\log\Omega}{N}\right)_{V,E,N=\overline{N}}
        \end{equation}
        Again using chain rule and \eqref{basic:thermo},
        \begin{equation}
            \zeta=\partialderivative{\log\Omega}{E}\derivative{E}{N}=-\frac{\mu}{kT}
        \end{equation}
        Note that at equilibrium, $T_1=T_2$, $P_1=P_2$, and $\mu_1=\mu_2$ as expected.
    \end{derivation}
    \item In S.I. Units, boltzmann's constant is $k\approx1.38\times10^{-23}$J/K. We will use natural units so $k=1$.
    \item Using basic calculus and the following lemma, the rest of thermodynamics can be derived
    \begin{lemma}
        If three variables are muturally related,
        \begin{equation}
            \left(\partialderivative{x}{y}\right)_z\left(\partialderivative{y}{z}\right)_x\left(\partialderivative{z}{x}\right)_y=-1
        \end{equation}
        "The proof is left as an exercise to the reader"---James Davis
    \end{lemma}
    The mathematics will not be shown here, but a few things should be noted
    \begin{itemize}
        \item Following \eqref{basic:thermo}, the intrinsic fields can be written as
        \begin{equation}
            P=-\left(\partialderivative{E}{V}\right)_{N,S}\:\:\:\:\:\:\mu=\left(\partialderivative{E}{N}\right)_{V,S}\:\:\:\:\:\:T=\left(\partialderivative{E}{S}\right)_{N,V}\label{intrinsic-fields}
        \end{equation}
        \item The Helmholtz free energy $A$, Gibbs free energy $G$ and enthalpy $H$ are given by
        \begin{align}
            A&=E-TS\\
            G&=A+PV=E-TS+PV=\mu N\\
            H&=E+PV=G+TS
        \end{align}
        \item Since the specific heat is $C=\derivative{Q}{T}=\frac{T\dd S}{\dd T}$, the specific heat at constant volume $C_V$ and the specific heat at constant pressure $C_P$ are
        \begin{equation}
            C_V=T\left(\partialderivative{S}{T}\right)_{N,V}=\left(\partialderivative{E}{T}\right)_{N,V}\label{cv}
        \end{equation}
        and
        \begin{equation}
            C_P=T\left(\partialderivative{S}{T}\right)_{N,P}=\left(\partialderivative{(E+PV)}{T}\right)_{N,P}=\left(\partialderivative{H}{T}\right)_{N,P}\label{cp}
        \end{equation}
    \end{itemize}
\end{itemize}
\subsection{Ideal Gas Law}
\subsubsection{Simple Derivation}
\begin{itemize}
    \item The assumption for the ideal gas is that the probability of finding any of the particles in any region of the available volume is independent of the locations of all other particles. Then, the total number of ways in which $N$ particles can be distributed is just equal to the product of the number of ways each particle is found in a given region, which is proportional to $V$. Thus,
    \begin{equation}
        \Omega(N,E,V)\propto V^N\label{ideal-gas:assumption}
    \end{equation}
    \begin{derivation}
        From \eqref{13}, and assuming a proportionality constant $c$ for \eqref{ideal-gas:assumption}
        \begin{equation}
            \frac{P}{T}=\left(\partialderivative{\log\Omega(N,E,V)}{V}\right)_{N,E}=\frac{\dd}{\dd V}(N\log V+\log c)=\frac{N}{V}\label{intermediate:ideal-gas-law}
        \end{equation}
        since $N=n N_A$ where $N_A$ is Avagadro's number, \eqref{intermediate:ideal-gas-law} can be rearranged as $PV=nRT$ where $R=N_A$ ($k=1$ is used). This holds for any classical system composed of noninteracting particles.
    \end{derivation}
\end{itemize}
\subsubsection{More General Quantum Mechanical Derivation}
\begin{itemize}
    \item Now model the system as a quantum-mechanical infinite well (3D) with side-lengths $L=V^{1/3}$. The bound state energies of the system (which can be derived by solving the time-independent Schrodinger equation) are
    \begin{equation}
        \epsilon(n_x,n_y,n_z)=\frac{\pi^2}{2mL^2}(n_x^2+n_y^2+n_x^2);\:\:\:\:\: n_x,n_y,n_z\in\mathbb{Z}_+
    \end{equation}
    For a single (non-relativistic) particle at a given energy, the number of microstates is equal to the number of independent energy eigenstates with that energy, i.e.
    \begin{equation}
        (n_x^2+n_y^2+n_z^2)=\frac{2mV^{2/3}\epsilon}{\pi^2}=\epsilon^*\label{27}
    \end{equation}
    The number of solutions to \eqref{27} is also $\Omega(1,\epsilon,V)$ since it's 1 particle, $\epsilon$ energy, and $V$ volume.
    \item This model can be extended further to suggest that $\Omega(N,E,V)$ may be equal to the number of independent solutions to
    \begin{equation}
        \sum_{r=1}^{3N}n_r^2=\frac{2mV^{2/3}E}{\pi^2}=E^*\label{28}
    \end{equation}
    for N independent (non-interacting) particles in an infinite well. The number of solutions or $\Omega(N,E,V)$ can only depend on $N$ and $E^*$. As $E^*$ is a constant multiple of $V^{2/3}E$, the $E$ and $V$ dependence of $\Omega(N,E,V)$ is $V^{2/3}E$. Therefore,
    \begin{equation}
        S(N,V,E)\equiv k\log\Omega(N,V,E)\equiv k\log\Omega(N,V^{2/3}E)\equiv S(N,V^{2/3}E)\label{29}
    \end{equation}
    \begin{derivation}
        For a \textit{reversible adiabatic process} where $S$ and $N$ are constant, $V^{2/3}E=c$ constant due to \eqref{29}. From \eqref{intrinsic-fields}
        \begin{equation}
            P=-\left(\partialderivative{E}{V}\right)_{N,S}=-\frac{\dd}{\dd V}cV^{-2/3}=\frac{2}{3}(cV^{-2/3})V^{-1}=\frac{2}{3}\frac{E}{V}\label{two-third:energy-density}
        \end{equation}
        The pressure of a system of non-relativistic, non-interacting particles is exactly $\frac{2}{3}$ of its energy density.
        
        Rearranging \eqref{two-third:energy-density}, $E=\frac{2}{3}PV$. Since $c=V^{2/3}E=V^{2/3}\frac{2}{3}PV\implies PV^{5/3}=\frac{3}{2}c$ constant.
    \end{derivation}
    \item For a specific value of $E^*$, the problem of finding integer solutions to \eqref{28} can be reframed as finding integer lattice points on the surface of a $3N$ dimensional hypersphere with radius $\sqrt{E^*}$. This is very difficult to count. So first, consider a simper case of counting all lattice points inside the "positive part" of the volume of the hypersphere. This can be seen as the number of microstates with energy less than or equal to $E$ (with the same $V$ of course). This volume will be called $\Sigma(E^*)$
    \item These formulas are given by mathematicians and \textbf{should not} be questioned.
    \begin{enumerate}
        \item The volume of a N dimensional hypersphere as a function of its radius is \begin{equation}
            V(N,R)=\frac{\pi^{n/2}}{(n/2)!}R^N\label{sphere}
        \end{equation}
        \item The \textit{Sterling Approximation} states that for big $n$ \begin{equation}
            \log (n!)\approx n\log n-n\label{sterling}
        \end{equation}
    \end{enumerate}
    \item Since the number of integer lattice points contained in a continuous region is approximately equal to the volume of the region itself, using \eqref{sphere}, and resubstituting the definition of $E^*$ from \eqref{28}
    \begin{equation}
        \Sigma(E^*)\approx\left(\frac{1}{2}\right)^{3N}\left(\frac{\pi^{3N/2}}{(3N/2)!}{E^*}^{3N/2}\right)=\frac{V^N}{(3N/2)!}\left(\frac{mE}{2\pi}\right)^{3N/2}\label{33}
    \end{equation}
    Note that the $\left(\frac{1}{2}\right)^{3N}$ comes from the fact that only the positive part of the hypersphere is considered. Now taking logarithm using \eqref{sterling}, \begin{equation}
        \log\Sigma(N,V,E)\approx N\log\left[V\left(\frac{mE}{2\pi}\right)^{3/2}\right]-\frac{3}{2}N\log\left(\frac{3}{2}N\right)+\frac{3}{2}N=N\log\left[V\left(\frac{mE}{3\pi N}\right)^{3/2}\right]+\frac{3}{2}N\label{34}
    \end{equation}
    \item Since the simpler problem is solved, we can tackle the original problem. Now the enemy (you) imposed a small $\epsilon<<E$ on the order $\epsilon/E=\mathcal{O}(1/\sqrt{N})$ where the number of microstates $\Omega(N,V,E)$ with total energy between $E-\frac{1}{2}\epsilon$ and $E+\frac{1}{2}\epsilon$ is to be found. Using local linear approximation,
    \begin{equation}
        \Omega(N,V,E)\approx\partialderivative{\Sigma(N,V,E)}{E}\bigg|_{E}\epsilon=\frac{3N}{2}\frac{V^N}{(3N/2)!}\left(\frac{m}{2\pi}\right)^{3N/2}E^{3N/2-1}\epsilon=\frac{3N\epsilon}{2E}\Sigma(N,V,E)
    \end{equation}
    Taking the logarithm and substituting \eqref{34},\begin{equation}
        \log\Omega(N,V,E)\approx \log\Sigma(N,V,E)+\log\left(\frac{3}{2}N\right)+\log(\frac{\epsilon}{E})
    \end{equation}
    \item The leading term $\log\Sigma(N,V,E)$ is $\mathcal{O}(N)$. The second term is trivially $\mathcal{O}(\log N)$. The final term is slightly more complicated. Since $\epsilon/E$ is $\mathcal{O}(N^{-1/2})$, its logarithm is likely negative. But more importantly, its magnitude is $\mathcal{O}(\log N)$. As $N$ becomes very large, only the leading term has a significant contribution; thus, \begin{equation}
        \log\Omega\approx\log\Sigma\approx N\log\left[V\left(\frac{mE}{3\pi N}\right)^{3/2}\right]+\frac{3}{2}N\label{37}
    \end{equation}
    \item Note two things. Firstly, the $\epsilon$ imposed by the enemy is insignificant. As long as it is similar to $\mathcal{O}(N^{-1/2})$, its contribution is insignificant relative to the other terms. Secondly, as $\log\Omega\approx\log\Sigma$ shows that the rate at which the number of microstates of the system increases as a function of energy is so great, that the microstates with energies $0$ to $E-\frac{1}{2}\epsilon$ are very few and far between compared to those with energies near $E$.
    \item With a good approximation for $\log\Omega$, the rest of the thermodynamics can be derived.
    \begin{derivation}
        First of all, from the definition of entropy \eqref{entropy},
        \begin{equation}
            S(N,V,E)=\log\Omega=N\log\left[V\left(\frac{mE}{3\pi N}\right)^{3/2}\right]+\frac{3}{2}N\label{wrong:entropy}
        \end{equation}
        Isolating the energy $E$,
        \begin{equation}
            E(S,N,V)=\frac{3\pi N}{mV^{2/3}}\exp(\frac{2S}{3N}-1)
        \end{equation}
        From \eqref{intrinsic-fields}, 
        \begin{equation}
            \left(\partialderivative{S}{E}\right)_{N,V}=\frac{1}{T}=\frac{3}{2}\frac{N}{E}
        \end{equation}
        \begin{equation}
            E=\frac{3}{2}NT=\frac{3}{2}nRT\label{41}
        \end{equation}
        Using \eqref{cv}, the specific heat at constant volume is
        \begin{equation}
            C_V=\left(\partialderivative{E}{T}\right)_{N,V}=\frac{3}{2}nR\label{ideal-gas:cv}
        \end{equation}
        From \eqref{two-third:energy-density}, 
        \begin{equation}
            P=\frac{2}{3}\frac{E}{V}=\frac{nRT}{V}\implies PV=nRT
        \end{equation}
        From \eqref{cp}, the specific heat at constant pressure is
        \begin{equation}
            C_P=\left(\partialderivative{H}{T}\right)_{N,V}=C_V+\frac{\dd}{\dd T}(PV)=\frac{3}{2}nR+nR=\frac{5}{2}nR\label{ideak-gas:cp}
        \end{equation}
        Then the ratio between the two specific heats is
        \begin{equation}
            \gamma\equiv\frac{C_P}{C_V}=\frac{5}{3}
        \end{equation}
    \end{derivation}
    \item Suppose the ideal gas undergoes an \textit{isothermal change} with constant $T$ and $N$. From \eqref{41}, the energy $E$ remains constant. Using logarithm laws, the entropy can be rewritten as
    \begin{equation}
        S(N,V,E)=N\log V+N\log\left[\left(\frac{mE}{3\pi N}\right)^{3/2}\right]+\frac{3}{2}N
    \end{equation}
    Since $T,N,E,m$ are all constants, the only term that changes is the leading term. Thus,
    \begin{equation}
        \Delta S=N\log\left(\frac{V_f}{V_i}\right)
    \end{equation}
\end{itemize}
\section{Gibb's Paradox and Correction}
\subsection{Entropy of Mixing}
\begin{itemize}
    \item Mixing two ideal gasses at the same temperature each with entropy
    \begin{equation}
        S_i=N_i\log V_i+\frac{3}{2}N_i\left[1+\log\left(\frac{m_iT}{2\pi}\right)\right]\:\:\:\:\:\:\:\:i=1,2
    \end{equation}
    After the mixing, the change in entropy (known as the \textit{entropy of mixing}) would be
    \begin{equation}
        \begin{aligned}
            \Delta S&=\sum_{i=1}^2N_i\log (V_1+V_2)+\frac{3}{2}N_i\left[1+\log\left(\frac{m_iT}{2\pi}\right)\right]-\sum_{i=1}^2N_i\log V_i+\frac{3}{2}N_i\left[1+\log\left(\frac{m_iT}{2\pi}\right)\right]\\
            &=N_1\log\left(\frac{V_1+V_2}{V_1}\right)+N_2\log\left(\frac{V_1+V_2}{V_2}\right)
        \end{aligned}
    \end{equation}
    \item For the case where the number densities of the two ideal gases are different, mixing should be an irreversible process. This is indeed true as $\Delta S>0$.
    \item If the number densities of the two ideal gases are equal, mixing should be reversible as the partition can be reinserted and nothing should have change. Thus, it is expected that $\Delta S=0$.
    However, if $V=\alpha N$,
    \begin{equation}
        \Delta S=N_1\log\left(\frac{N_1+N_2}{N_1}\right)+N_2\log\left(\frac{N_1+N_2}{N_2}\right)>0\label{paradox}
    \end{equation}
    This suggests that there must be an omission that has occured.
    \item Approximating $\log (n!)\approx n\log n-n\approx n log n$ for big $n$, \eqref{paradox} can be simplified to
    \begin{equation}
        \Delta S=S_f-S_1-S_2\approx\log [(N_1+N_2)!]-\log(N_1!)-\log(N_2!)
    \end{equation}
    From the simplified expression, it seems like executing $S\:-=\:\log(N!);$ would resolve this issue (challenge, sorry praxis). This is the solution recommended by Gibbs.
    \item With this modification,
    \begin{align}
        S(N,V,E)&=N\log\left[\frac{V}{N}\left(\frac{mE}{3\pi N}\right)^{3/2}\right]+\frac{5}{2}N\label{ideal-gas:entropy}\\ 
        &=N\log\left(\frac{V}{N}\right)+\frac{3}{2}N\left[\frac{5}{3}+\log(\frac{mT}{2\pi})\right]
    \end{align}
    \item This also resolves another issue. The entropy of the ideal gas derived in \eqref{wrong:entropy} does not behave like an extensive property of the system. If all intensive variables are held constant and the size of the system is doubled, the entropy should also double. 
    \begin{equation}
        \begin{aligned}
            S(2N,2V,2E)-2S(N,V,E)&=\left[(2N)\log\left[(2V)\left(\frac{m(2E)}{3\pi(2N)}\right)^{3/2}\right]+\frac{3}{2}(2N)\right]-\left[2N\log\left[V\left(\frac{mE}{3\pi N}\right)^{3/2}\right]+3N\right]
            \\&=2N\log 2
        \end{aligned}
    \end{equation}
    With the modification, entropy described by \eqref{ideal-gas:entropy} does behave like an extensive field!
    \begin{equation}
        S(\alpha N,\alpha V, \alpha E)=\alpha S(N,V,E)
    \end{equation}
    \item The corrected entropy of mixing for the same gas would be
    \begin{equation}
        \Delta S=S_F-S_1-S_2=(N_1+N_2)\log\left(\frac{V_1+V_2}{N_1+N_2}\right)-N_1\log\left(\frac{V_1}{N_1}\right)-N_2\log\left(\frac{V_2}{N_2}\right)\label{mixing:same}
    \end{equation}
    Here when $N_1/V_1=N_2/V_2$, $\Delta S$ is indeed zero!
    \item For two different gasses,
    \begin{equation}
        \Delta S=S_{1F}+S_{2F}-S_1-S_2=N_1\log\left(\frac{V_1+V_2}{V_1}\right)+N_2\left(\frac{V_1+V_2}{V_2}\right)\label{mixing:different}
    \end{equation}
    Note that this is different, as the total final entropy of two different gasses is the sum of the entropies of each gas.
\end{itemize}
\subsection{More on the Ideal Gas}
\begin{itemize}
    \item The quantities related to the chemical potential can now be derived as the correction is nessecary to derive the correct result. 
    \begin{derivation}
        With the correction, the energy must be rederived. Firstly, isolating for the energy,
        \begin{equation}
            E(S,N,V)=\frac{3\pi N^{5/3}}{mV^{2/3}}\exp\left(\frac{2S}{3N}-\frac{5}{2}\right)
        \end{equation}
        The chemical potential can also be derived
        \begin{equation}
            \mu=\left(\partialderivative{E}{N}\right)_{V,S}=E\left(\frac{5}{3N}-\frac{2S}{3N^2}\right)
        \end{equation}
        From equations \eqref{two-third:energy-density} and \eqref{41},
        \begin{equation}
            \mu=\frac{1}{N}\left(E+\frac{2}{3}E-\frac{2S}{3N}\right)=\frac{1}{N}(E+PV-TS)\equiv\frac{G}{N}
        \end{equation}
        In terms of the variables $N,V,T$, the chemical potential is expressed as
        \begin{equation}
            \mu(N,V,T)=T\log\left[\frac{N}{V}\left(\frac{2\pi}{mT}\right)^{3/2}\right]
        \end{equation}
        The Helmholtz free neergy is 
        \begin{equation}
            A=E-TS=G-PV=N(\mu-T)=NT\left[\log\left[\frac{N}{V}\left(\frac{2\pi}{mT}\right)^{3/2}\right]-1\right]
        \end{equation}
    \end{derivation}
    \item Note that the Helmholtz free energy is an extensive property whereas the chemical potential is an intensive property.
\end{itemize}
\subsection{Indistinguishable Particles}
\begin{itemize}
    \item The reason that the original derivation of the entropy of the system of $N$ non-interacting particles requires the correction is because the derivation assumes that the particles are distinguishable. 
    \item In reality, the most particles are identical. This means that for 2 particles, the state with $\ket{1}\otimes\ket{2}$ and $\ket{2}\otimes\ket{1}$ are indistinguishable and only considered as 1 microstate. Therefore, the original derivation overcounted the number of microstates (by $N!$). 
    \item The correct way to represent a system is to represent its particles distribution over energies by numbers.
    \begin{equation}
        E=\sum_{i=0}^\infty n_i\epsilon_i
    \end{equation}
    where $n_0,n_1,n_2\dots$ represent the number of particles in the system with energy $\epsilon_0,\epsilon_1,\epsilon_2\dots$ respectively.
    \item The total number of ways to permute a specific microstate of a system of indistinguishable particles is then $N!/\prod n_i!$ as there are $N!$ ways of assigning each particle to each energy and $\prod n_i!$ overcounted due to particles with the same energy $\epsilon_i$.
    \item The correction done by Gibbs (of $N!$) is assuming that no two particles are at the exact same energy, which is true in the classical limit where $\langle n_i\rangle<<1$, which occurs when the system has either a sufficiently high temperature or low density.
    \item Define the \textit{weight factor}
    \begin{equation}
        w\{n_i\}=\frac{1}{\prod_in_i!}
    \end{equation}
    As mentioned before, in the classical limit $w\{n_i\}=1$. 
\end{itemize}
\section{Chapter 1 Problems}
\begin{sol}[1.1]
    \begin{enumerate}[label=\textbf{(\alph*)}]
        \item Take the taylor expansion of $\log\Omega^0(E^0,E_1)=\log\Omega_1(E_1)+\log\Omega_2(E_2)$ at equilibrium energy $E_1=\overline{E_1}$.
        \begin{equation}
            \begin{aligned}
                \log\Omega^0(E_1)&=\log\Omega_1(\overline{E_1})+\log\Omega_2(\overline{E_2})+\left(\partialderivative{\log\Omega_1}{E_1}\bigg|_{E_1=\overline{E_1}}+\partialderivative{\log\Omega_2}{E_2}\bigg|_{E_2=\overline{E_2}}\derivative{E_2}{E_1}\right)(E_1-\overline{E_1})\\
                &+\frac{1}{2}\left(\partialderivative{^2\log\Omega_1}{E_1^2}\bigg|_{E_1=\overline{E_1}}+\partialderivative{^2\log\Omega_2}{E_2^2}\bigg|_{E_2=\overline{E_2}}\left(\derivative{E_2}{E_1}\right)^2+\partialderivative{\log\Omega_2}{E_2}\bigg|_{E_2=\overline{E_2}}\derivative{^2E_2}{E_1^2}\right)(E_1-\overline{E_1})^2\\
                &+\mathcal{O}\left[(E_1-\overline{E_1})^3\right]
            \end{aligned}
        \end{equation}
        The linear term of the series is zero as the taylor expansion is taken at equilibrium energies (see \eqref{equilibrium-conducting} and \eqref{equilibrium-conducting-2}). In the quadratic term, the third contribution is also zero as the 2nd derivative of $E_2$ is zero. For the remaining part of the quadratic term,
        \begin{equation}
            \partialderivative{^2\log\Omega}{E^2}\bigg|_{E=\overline{E}}=\partialderivative{}{E}\frac{1}{T}=\partialderivative{T}{E}\partialderivative{}{T}\frac{1}{T}=-\frac{1}{C_VT^2}
        \end{equation}
        Define
        \begin{equation}
            \frac{1}{a^2}\equiv\frac{1}{C_{V1}T^2}+\frac{1}{C_{V2}T^2}\label{rms:1-1}
        \end{equation}
        Simplifying,
        \begin{equation}
            \log\Omega^0(E_1)=\log\Omega^0(\overline{E_1})-\frac{(E_1-\overline E_1)^2}{2a^2}+\mathcal{O}\left[(E_1-\overline{E_1})^3\right]\label{gaussian:1-1}
        \end{equation}
        The $\mathcal{O}\left[(E_1-\overline{E_1})^3\right]$ term can be neglected since when the systems become large, the temperature becomes less dependent on the energy.
        Exponentiating both sides of \eqref{gaussian:1-1},
        \begin{equation}
            \Omega^0(E_1)=\Omega^0(\overline{E_1})\exp\left(-\frac{(E_1-\overline{E_1})^2}{2a^2}\right)
        \end{equation}
        Thus, the root-mean-square deviation of $E_1$ is $a$ in \eqref{rms:1-1}.
        \item For the ideal gas, $C_V=\frac{3}{2}N$ from \eqref{ideal-gas:cv}. Thus, 
        \begin{align}
            \frac{1}{a^2}&=\frac{1}{N_1T^2}+\frac{1}{N_2T^2}\\
            a&=T\sqrt{\frac{N_1+N_2}{N_1N_2}}
        \end{align}
        Where $T$ is the equilibrium temperature.
    \end{enumerate}
\end{sol}
\begin{sol}[1.2]
    The total entropy $S$ and number of microstates and $\Omega$ are that for $n$ systems with entropies $S_1,S_2\dots S_n$ and number of microstates $\Omega_1,\Omega_2\dots\Omega_n$,
    \begin{align}
        S&=\sum_{j=1}^n S_i\:\:\:\:\:\:\:\:\Omega=\prod_{j=1}^n\Omega_i\\
        S&=f(\Omega)=\sum_{j=1}^n f(\Omega_i)
    \end{align}
    Combining the definitions,
    \begin{equation}
        f\left(\prod_{j=1}^n\Omega_i\right)=\sum_{j=1}^nf(\Omega_i)
    \end{equation}
    It is shown by Dieudonné, Jean in \textit{Foundations of Modern Analysis} (1969) pp. 84 that the only functions satisfying this definition are logarithmic functions.
\end{sol}
\begin{sol}[1.7]
    Remember $\hbar=c=1$. We can rearrange the equation
    \begin{equation}
        n_x^2+n_y^2+n_z^2=\frac{L^2}{2\pi^2}\epsilon^2
    \end{equation}
    \begin{lemma}
        For $N>1$ random variables $\{r_1,r_2\dots r_N\}$ that has a uniform distribution within the interval $[0,M)$, both
        \begin{equation}
            \frac{3}{4}N<\left\langle\frac{(\sum_{n=1}^Nr_n)^2}{\sum_{n=1}^Nr_n^2}\right\rangle<N
        \end{equation}
        \begin{equation}
            \lim_{N\to\infty}\frac{1}{N}\left\langle\frac{(\sum_{n=1}^Nr_n)^2}{\sum_{n=1}^Nr_n^2}\right\rangle=\frac{3}{4}
        \end{equation}
    \end{lemma}
    \begin{prooof}
        Prof. Davis will have a proof.
    \end{prooof}
    Let $N>>1$ particles, there will be $3N$ quantum numbers whose squares add up to some number. From the theorem, 
    \begin{equation}
        \sum_{i=1}^{3N}n_i^2=\frac{L^2}{2\pi^2}\sum_{i=1}^{3N}\epsilon_i^2=\frac{L^2}{2\pi^2}\frac{4}{3N}\left(\sum_{i=1}^{3N}\epsilon_i\right)^2=\frac{2L^2E^2}{3\pi^2N}\equiv R^2
    \end{equation}
    As it is shown in \eqref{34}, the number of solutions that is on the surface on the sphere (assuming $\epsilon/E=\mathcal{O}(1/\sqrt{N})$) is approximately equal to the volume of positive part the $3N$ dimensional sphere of radius $R$. Assuming the particles are distinguishable, this is also the number of microstates of the system.
    \begin{align}
        \Omega(N,V,E)&\approx \left(\frac{1}{2}\right)^{3N}\frac{1}{(3N/2)!}\frac{V^{N/2}E^{3N/2}}{N^{3N/4}}\\
        S=\log \Omega&\approx \frac{N}{2}\log V+\frac{3N}{2}\log\frac{E}{4}-\left[\frac{3N}{2}\log\left(\frac{3N}{2}\right)-\frac{3N}{2}\right]-\frac{3N}{2}\log \sqrt N\\
        &=\frac{N}{2}\log V+\frac{3N}{2}\left[1+\log\left(\frac{E}{6N^{3/2}}\right)\right]
    \end{align}
    The temperature is
    \begin{equation}
        \left(\partialderivative{S}{E}\right)_{N,V}=\frac{1}{T}=\frac{3N}{2E}
    \end{equation}
    Isolating $E$ yields
    \begin{equation}
        E=\frac{6N^{3/2}}{V^{1/3}}\exp\left(\frac{2S}{3N}-1\right)=\frac{3}{2}NT
    \end{equation}
    The pressure is 
    \begin{equation}
        \left(\partialderivative{S}{V}\right)_{N,E}=\frac{N}{2V}=\frac{P}{T}
    \end{equation}
    Calculating the heat capacities,
    \begin{align}
        C_V&=\left(\partialderivative{E}{T}\right)_{N,V}=\frac{3}{2}N\\
        C_P&=\left(\partialderivative{H}{T}\right)_{N,P}=2N
    \end{align}
    \begin{equation}
        \therefore \gamma=\frac{C_P}{C_V}=\frac{4}{3}
    \end{equation}
\end{sol}
\begin{sol}[1.11]
    Let nitrogen be gas 1 and oxygen be gas 2. Assume the they are ideal gasses,
    \begin{equation}
        V_1=\frac{n_1RT}{P}=\frac{(\SI{4}{mol})(\SI{82.057}{cc.atm K.^{-1}mol.^{-1}})(\SI{300}{K})}{(\SI{1}{atm})}=\SI{98468}{cc}
    \end{equation}
    \begin{equation}
        V_2=\frac{n_2RT}{P}=\frac{(\SI{1}{mol})(\SI{82.057}{cc.atm K.^{-1}mol.^{-1}})(\SI{300}{K})}{(\SI{1}{atm})}=\SI{24617}{cc}
    \end{equation}
    Note that $V_1+V_2=\SI{123086}{cc}$. From equation \eqref{mixing:different}, the entropy of mixing of two gasses is
    \begin{equation}
        \Delta S=N_1\log\left(\frac{V_1+V_2}{V_1}\right)+N_2\log(\frac{V_1+V_2}{V_2})
    \end{equation}
    As $nR=N$, (R is also \SI{8.314e7}{erg.K.^{-1}mol.^{-1}} in proper units) and dividing by \SI{5}{mol} of air that is created by the mixing,
    \begin{equation}
        \begin{aligned}
            \Delta S&=\frac{1}{5}(\SI{8.314e7}{erg.K.^{-1}mol.^{-1}})\left[(\SI{4}{mol})\log(\frac{\SI{123086}{cc}}{\SI{98468}{cc}})+(\SI{1}{mol})\log(\frac{\SI{123086}{cc}}{\SI{24617}{cc}})\right]
            \\&=\SI{4.16e7}{erg.K^{-1}.mol^{-1}}
        \end{aligned}
    \end{equation}
    There is also a much simpler calculation where $V$ is substituted for $nRT/P$ with $RT/P$ cancelling out.
    \begin{equation}
        \Delta S=\frac{1}{5}(\SI{8.314e7}{erg.K.^{-1}mol.^{-1}})\left[(\SI{4}{mol})\log(\frac{\SI{5}{mol}}{\SI{4}{mol}})+(\SI{1}{mol})\log(\frac{\SI{5}{mol}}{\SI{1}{mol}})\right]
            =\SI{4.16e7}{erg.K^{-1}.mol^{-1}}
    \end{equation}
    But given how much advanced mathematics is done, one needs one's daily dose of number-krunching.
\end{sol}
\begin{sol}[1.13]
    If the initial temperatures are different, the total energy in the two gases would be 
    \begin{equation}
        E=E_1+E_2=\frac{3}{2}(N_1T_1+N_2T_2)\equiv\frac{3}{2}(N_1+N_2)T_{eq}
    \end{equation}
    Firstly consider the case of two identical gases. The entropy of mixing would be
    \begin{align}
        \Delta S=S_F-S_1-S_2&=\Delta S_0+\frac{3}{2}(N_1+N_2)\log\left(\frac{E}{3(N_1+N_2)}\right)-\frac{3}{2}N_1\log\left(\frac{T_1}{2}\right)-\frac{3}{2}N_2\log\left(\frac{T_2}{2}\right)\\
        &=\Delta S_0+\frac{3}{2}\left[N_1\log\left(\frac{N_1+\beta N_2}{N_1+N_2}\right)+N_2\log\left(\frac{\beta^{-1}N_1+N_2}{N_1+N_2}\right)\right]\bigspace \beta\equiv\frac{T_2}{T_1}\\
        &=\Delta S_0+\frac{3}{2}\left[N_1\log\left(\frac{T_{eq}}{T_1}\right)+N_2\log\left(\frac{T_{eq}}{T_2}\right)\right]
    \end{align}
    where $\Delta S_0$ is the entropy of mixing if the initial temperatures are the same. 

    Next consider the case of two different gasses. Then, the entropy of mixing would be 
    \begin{align}
        \Delta S=S_{1F}+S_{2F}-S_{1}-S_2&=\Delta S_0+\frac{3}{2}N_1\log\left(\frac{T_{eq}}{2}\right)+\frac{3}{2}N_2\log\left(\frac{T_{eq}}{2}\right)-\frac{3}{2}N_1\log\left(\frac{T_{1}}{2}\right)-\frac{3}{2}N_2\log\left(\frac{T_{2}}{2}\right)\\
        &=\Delta S_0+\frac{3}{2}\left[N_1\log\left(\frac{T_{eq}}{T_1}\right)+N_2\log\left(\frac{T_{eq}}{T_2}\right)\right]
    \end{align}
    As expected since temperature is an intensive field, the contribution to the entropy of mixing from the different difference in temperature is independent of whether if two identical or different gases are mixed. However, the total entropy of mixing still depends on whether the gases are identical or not as $\Delta S_0$ would be different. 
\end{sol}
\section{Phase Space}
\begin{itemize}
    \item Ensemble theory considers the time-averaged behavior of a given system that can be in any one of the large number of distinct microstate for a given macrostate using expectation values.
    \item A classical system is fully specified by the positions and momenta of all its particles. Therefore, for $N$ particles it requires $3N$ generalized position $q_i$ and generalized momenta $p_i$. These $6N$ coordinates $q_1,q_2\dots q_{3N},p_1,p_2\dots p_{3N}$ is a point in a $6N$ dimensional phase space.
    \item The time evolution of the system (in the Hamiltonian formalism) is goverened by the canonical equations of motion 
    \begin{align}
        \dot q_i&=\partialderivative{H}{p_i}\label{position}\\
        \dot p_i&=-\partialderivative{H}{q_i}\label{momentum}
    \end{align}
    \item As time passes, the coordinates which defines the microstate of the system also changes. Define,
    \begin{equation}
        \ve q\equiv (q_1,q_2\dots q_{3N}),\smallspace \ve p\equiv p_1,p_2\dots p_{3N}
    \end{equation}
    In phase space, the point $(\ve q,\ve p)$ moves in a \textit{trajectory} and its speed and direction is determined by the \textit{velocity vector} $\mathbf{v}\equiv (\dot{\ve q},\dot{\ve p})$
    \item The finite volume $V$ restricts the coordinates $q_i$ and the finite energy $E$ limits the values of both the positions $q_i$ and momenta $p_i$ (as it is related to the Hamiltonian).
    \item If the total energy of the system is know to have a precise value, the trajectory of the system will be constrained to a $6N-1$ dimensional hypersurface. If the total energy is constrained within $E-\frac{1}{2}\epsilon$ and $E+\frac{1}{2}\epsilon$, the trajectory would be restricted to a shell about the hypersurface.
    \item The \textit{density function} $\rho(\ve q,\ve p;t)$ is a function such that $\rho(\ve q,\ve p;t)\dd^{3N}\ve q\dd^{3N}\ve p$ is the probability of finding the system (macrostate) at the coordinates $(\ve q,\ve p)$ at a given point in time $t$. 
    \item Let it be normalized. It's like $\rho(\ve q,\ve p;t)=|\braket{\ve q,\ve p}{\psi,t}|^2$ in a way (for state vector representing the system $\ket{\psi,t}$ and state vector microstate $\ket{\ve q,\ve p}$). Of course, that would require conservation of probability (which will be proved later). 
    \item In classical mechanics, all observables are a function of the positions and momenta (similar to operators on the state vector in quantum). The expectation value of an observable $f(\ve q,\ve p)$ is
    \begin{equation}
        \langle f\rangle=\int f(\ve q,\ve p)\rho(\ve q,\ve p;t)\dd^{3N}\ve q\dd^{3N}\ve p
    \end{equation}
\end{itemize}
\begin{definition}
    An ensemble is \textbf{stationary} iff
    \begin{equation}
        \partialderivative{\rho}{t}=0\:\forall t
    \end{equation}
    For an stationary ensemble, the expectation values for all observables are time-independent. Therefore, they represent systems in \textbf{equilibrium}.
\end{definition}
\subsection{Conservation of Probability}
\begin{proposition}
    Let the probability current be $\ve j\equiv \rho\ve v$. Just like in quantum mechanics the probability is conserved.
    \begin{equation}
        \partialderivative{\rho}{t}+\nabla\cdot\ve j=0\label{conservation:probability}
    \end{equation}
\end{proposition}
\begin{prooof}
    Consider any closed region $\omega$ within phase space and let the bounding surface be $\sigma$. Then, the change in probability (over time) of finding in microstates represented by points within the region (or the probability entering the region) is 
    \begin{equation}
        \partialderivative{}{t}\int_\omega \rho\:\dd\omega
    \end{equation}
    Over a surface element $\dd\sigma$, the probability exiting (the flux) is the velocity in the direction of the normal multiplied by the density at that point $\rho\:\ve v\cdot\ve{\hat{n}}\:\dd\sigma$. Integrating over the entire surface yields the total probability exiting the region, which must add up with the probability entering the region to zero.
    \begin{equation}
        \partialderivative{}{t}\int_\omega\rho\:\dd\omega+\oint_\sigma \rho\:\ve v\cdot\ve{\hat{n}}\:\dd\sigma=0
    \end{equation}
    Applying divergence theorem on the second term yields
    \begin{equation}
        \partialderivative{}{t}\int_\omega\rho\:\dd\omega+\int_\omega\nabla\cdot(\rho\ve v)\:\dd\omega=0
    \end{equation}
    Differentiating under the integral sign and substituting the definition of $\ve j$ would yield
    \begin{equation}
        \int_\omega\partialderivative{\rho}{t}+\nabla\cdot\ve j\:\dd\omega=0
    \end{equation}
    Since this is true for every closed region $\omega$, the integrand must be zero. Therefore, probability is conserved. QED.
\end{prooof}
\subsection{Liouville's Theorem}
\begin{theorem}
    For two observables $A(\ve q,\ve p)$ and $B(\ve q,\ve p)$, define the poission bracket (\textbf{not} the commutator) as 
    \begin{equation}
        [A,B]=\partialderivative{A}{q_i}\partialderivative{B}{p_i}-\partialderivative{A}{p_i}\partialderivative{B}{q_i}\label{poission:bracket}
    \end{equation}
    Note that the Einstein sum convenction is used in \eqref{poission:bracket}. Then,
    \begin{equation}
        \derivative{\rho}{t}=\partialderivative{\rho}{t}+[\rho,H]=0
    \end{equation}
\end{theorem}
\begin{prooof}
    \textbf{Note that Einstein sum convention is used in this proof. YOU CAN'T STOP ME.}

    From conservation of probability \eqref{conservation:probability}, the divergence can be expanded
    \begin{align}
        0&=\partialderivative{\rho}{t}+\partialderivative{\rho \dot q_i}{q_i}+\partialderivative{\rho \dot p_i}{p_i}\\
        &=\partialderivative{\rho}{t}+\partialderivative{\rho}{q_i}\dot q_i+\partialderivative{\rho}{p_i}\dot p_i+\rho\left(\partialderivative{\dot q_i}{q_i}+\partialderivative{\dot p_i}{p_i}\right)\label{111}
    \end{align}
    From the canonical equations of motion \eqref{position} and \eqref{momentum}, 
    \begin{align}
        \partialderivative{\dot q_i}{q_i}&=\partialderivative{}{q_i}\partialderivative{H}{p_i}=\partialderivative{^2H}{q_ip_i}\\
        \partialderivative{\dot p_i}{p_i}&=\partialderivative{}{p_i}\left(-\partialderivative{H}{q_i}\right)=-\partialderivative{^2H}{p_iq_i}
    \end{align}
    This shows that final term is zero. As $\rho$ is a function of variables $\ve q,\ve p, t$, the first three terms of \eqref{111} represents the total derivative with respect to time $\dd\rho/\dd t$. Now, substitute $\dot q_i$ and $\dot p_i$ using the canonical equations of motion.
    \begin{align}
        0&=\partialderivative{\rho}{t}+\partialderivative{\rho}{q_i}\partialderivative{H}{p_i}+\partialderivative{\rho}{p_i}\left(-\partialderivative{H}{q_i}\right)\\
        &=\partialderivative{\rho}{t}+[\rho,H]
    \end{align}
    QED.
\end{prooof}
\begin{itemize}
    \item For a stationary ensemble, Liouville's theorem requires $[\rho,H]=0$.
    \item The trivial solutions are $\rho(\ve q, \ve p)=$ constant. This corresponds to density functions that are independent of the coordinates hence uniform over phase space at all times. This means all possible microstates are equally likely, which also supports the axiom of "equal \textit{a priori} probabilities".
    \item Let $\omega$ denote the allowed region in phase space. For this case, the expectation value for any observable $f(\ve q,\ve p)$ is just
    \begin{equation}
        \langle f\rangle = \frac{1}{\omega}\int_{\omega}f(\ve q,\ve p)\dd\omega
    \end{equation}
    \item The ensemble resulting from these solutions is called the \textbf{microcanonical ensemble}.
    \item A more general solution to $[\rho,H]=0$ is 
    \begin{equation}
        \rho(\ve q, \ve p)=\rho(H(\ve q, \ve p))\propto\exp\left(-\frac{H(\ve q,\ve p)}{kT}\right)
    \end{equation}
    The class of ensembles that are proportional to the exponential function of the Hamiltonian is called the \textit{canonical ensemble}.
\end{itemize}
\section{Microcanonical Ensemble}
\begin{definition}
    The microcanonical ensemble is an ensemble where the macrostate of the system is defined by the number of molecules $N$, the volume $V$, and the energy $E$.
\end{definition}
\begin{itemize}
    \item 
\end{itemize}
\section{Chapter 2 Problems}
\end{document}