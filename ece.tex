\documentclass{article}
\usepackage[a4paper, margin=0.5in]{geometry}
\usepackage{amsmath}
\usepackage{amssymb}
\usepackage{physics}
\usepackage[american]{circuitikz}
\usepackage{physoly}

\begin{document}
\section{Lecture 1}
\subsection{An Electric Circult}
    An electric circult is an interconnection of circuit elements (incl. Conductors, semi-conductors, non-conductors). 

\subsection{Electrical Variables}
    To help us analyze electric circuits, we define several electrical variables:
    \begin{definition}
        Electric currents: The "movement" or rate of change of electrical charge. 
        \begin{equation}
            i\equiv\frac{dq}{dt}
        \end{equation} 
        
        \begin{enumerate}
            \item S.I. unit: Ampere = Columb per second.
            \item The current has a direction, and its direction is defined as the direction of positive charge. (Can be shown with $\leftarrow$ or $\rightarrow$)
        \end{enumerate}
    \end{definition}
    When performing circuit analysis, the direction of positive current is either given or we are required to guess a direction of positive current and verify it later. A positive current means the direction you guessed is correct. A negative current means the actual direction is the direction opposiing the current.
    \begin{definition}
        Voltage: The energy required for 1C of charge between point A and B in a circuit is called the voltage between points A and B. 
        \begin{equation}
            V\equiv\frac{dw}{dq}
        \end{equation} 
        \begin{enumerate}
            \item S.I. unit: Volt = Joule per columb.
            \item The voltage also have polarity ($+$ or $-$). Positive polarity means energy is consumed when the charge moves from A to B.
        \end{enumerate}
    \end{definition}
    If the polarity is not given, guess an polarity and a positive voltage indicates the guess is correct and a negative voltage indicates the guess is incorrect. 
    \begin{definition}
        Power: The rate of delivering or absorbing energy.
        \begin{equation}
            p\equiv\frac{dw}{dt}
        \end{equation}
        \begin{enumerate}
            \item S.I. unit: Watt = Joule per second.
            \item Using chain rule, 
            \begin{equation}
                p=\frac{dw}{dt}=\frac{dw}{dq}\frac{dq}{dt}=iv
            \end{equation}
        \end{enumerate}
    \end{definition}
\textbf{Note:} When a problem asks for a current or voltage, the direction/polarity \textbf{must} be indicated otherwise full credit will not be given.
\subsection{Passive Sign Convention}
\begin{definition}
    For a pair of v and i, PSC holds if the current direction \textbf{enters} the positive side of voltage polarity. If PSC holds, $p=+vi$; else $p=-vi$ 
    \begin{center}
        \begin{circuitikz}
            \draw (0,0) to[generic,o-o](5,0);
            \draw (7,0) to[generic,o-o](12,0);
            \draw (1.5,0.2) node{$+$};
            \draw (3.5,0.2) node{$-$};
            \draw (8.5,0.2) node{$+$};
            \draw (10.5,0.2) node{$-$};
            \draw[-latex](0.5,0)--(1,0)node[below]{$i_1$};
            \draw[-latex](8.5,0)--(8,0)node[below]{$i_2$};
        \end{circuitikz}
    \end{center}
    PSC holds on the left and does not hold on the right.
\end{definition}
\noindent The consequences of this convention are
\begin{enumerate}
    \item $p>0\implies$power is absorbed
    \item $p<0\implies$power is delivered (generated by the circuit element)
\end{enumerate}        

\section{Lecture 2}
\begin{example}[2]
    Identify the devices that generate power in the circuit below and find the unknowns in the table.
    \begin{center}
        \begin{circuitikz}
            \draw 
            (0,0) to[generic,V=$v_1$] (0,4) 
            to[generic,V=$v_3$] (6,4);
            \draw
            (0,0) to (6,0)
            to[generic,V=$v_4$] (6,4);
            \draw
            (6,0) to (12,0) 
            to[generic,V=$v_5$] (12,4)
            to (6,4);
            \draw
            (0,4) to[generic,V=$v_2$] (6,0);
        \end{circuitikz}
    \end{center}
\end{example}
\begin{theorem}
    \textbf{Conservation of Power} states that the algebraic sum of the power of all elements in a circuit is zero. (algebraic means the signs are preserved)
    \begin{equation}
        \sum p=0
    \end{equation}
    \begin{proof}
        Will not be presented at this point in the course.
    \end{proof}
\end{theorem}

\begin{example}[2]
    \begin{center}
        \begin{circuitikz}
            \draw
            (0,0)node[anchor=east]{Cir.1} to[generic,o-o] (8,0)node[anchor=west]{Cir.2};
            \draw[-latex] (1,0)--(2,0) node[below]{i(t)};
        \end{circuitikz}
    \end{center}
    Given the above circuit and
    \begin{align}
        v(t)&=50(1-e^{-5000t})\text{V}\\
        i(t)&=10e^{-5000t}\text{A}
    \end{align}
    Find the total energy transfered to this device after $t=0$.
\end{example}
\begin{sol}
    PSC holds. Thus,
    \begin{align}
        p(t)&=v(t)i(t)\\
        p(t)&=50(1-e^{-5000t})10(e^{-5000t})\text{W}
    \end{align}
    From the definition of power,
    \begin{align}
        p(t)&=\frac{dw}{dt}\therefore dw=p(t)dt\\
        w&=\int_0^\infty p(t)\dd t=\int_0^\infty 50(1-e^{-5000t})10(e^{-5000t}) \dd t=\frac{1}{20}\text{J}=50\text{mJ}
    \end{align}
\end{sol}
\section{Lecture 3}
\subsection{Circuit Elements}
\begin{enumerate}
    \item Independent sources:
    \begin{enumerate}
        \item \textit{Independent voltage sources:} A circuit element that has a specific voltage independent of the current that flows through it.
        
        For example, the voltage can be a fixed voltage like $v_s=2$V or a variable voltage source like $v_s=2\sin(50t+2)$.

        On diagrams, generic voltage source is shown on the left and fixed (DC) voltage sources is shown on the right. For DC sources, the uppercase letter $V$ is commonly used.
        \begin{center}
            \begin{circuitikz}
                \draw
                (0,0)to[generic,V=,o-o](4,0);
                \draw
                (6,0)node[below]{$+$} to[battery,o-o](10,0)node[below]{$-$};    
            \end{circuitikz}
        \end{center}
        For DC source, the longer lines denotes positive voltage polarity and shorter lines denote negative voltage polarity.

        An sinsudal voltage is drawn as
        \begin{center}
            \begin{circuitikz}
                \draw
                (0,0)to[sinusoidal voltage source,V=,o-o](6,0);
            \end{circuitikz}
        \end{center}

        Questions
        \begin{enumerate}
            \item What is the direction of $i$ for the following voltage source?
            \begin{center}
                \begin{circuitikz}
                    \draw
                    (0,0)to[generic,V=2V,o-o](6,0);
                \end{circuitikz}
            \end{center}
            Answer: You cannot determine.
            \item Does a voltage source always generate power?
            
            Answer: No. Take the voltage source above. Take a current $i_1=2\text{A}\rightarrow$. Then, PSC holds and $p=vi=(2\text{V})(2\text{A})=4\text{W}$. Since $p>0$, the voltage source absorbs power.
        \end{enumerate}

        \item \textit{Independent current source:} It gives a specific current independent of the voltage across it. Can be constant like $i_s=5$A or time dependent like $i_s=10\cos(15t)$
        
        The independent current source is drawn as
        \begin{center}
            \begin{circuitikz}
                \draw
                (0,0)to[I](6,0);
            \end{circuitikz}
        \end{center}

        Questions
        \begin{enumerate}
            \item What is the polarity of voltage across a current source?
            \item Does a current source always generate power?
        \end{enumerate}
        Do not let the word \textit{source} decieve you. A source is not associated with generating power.

        \item Dependent Sources:
        \begin{enumerate}
            \item \textit{Voltage-dependent voltage source:}
            \begin{center}
                \begin{circuitikz}
                    \draw
                    (0,0)to[cV,cV=$Kv_x$,o-o](6,0);
                \end{circuitikz}
            \end{center}
            The voltage of this source depends on voltages somewhere else in the circuit. i.e. $v_s=Kv_s$
            
            \item \textit{Current-dependent voltage source:}
            \begin{center}
                \begin{circuitikz}
                    \draw
                    (0,0)to[cI,cI=$Ki_x$,o-o](6,0);
                \end{circuitikz}
            \end{center}
            The voltage of this source depends on the current somewhere else in a circuit i.e. $v_s=k i_x$. Note k has the dimensions Current/Voltage.
            \item \textit{Voltage-dependent current source:}
            \item \textit{Current-dependent current source:}
        \end{enumerate}
    \end{enumerate}
\end{enumerate}
\section{Lecture 4}
\subsection{Circuit Elements (cont.)}
\begin{enumerate}
    \item Resistor: A circuit element that has keeps a constant ratio between a voltage and current, such that
    \begin{equation}
        R\equiv\frac{v}{i}
    \end{equation}
    S.I. unit: Volt per Amp or Ohm ($\Omega$).

    The resistor is drawn like this. If PSC holds, $v=Ri$ else $v=-Ri$. This is also known as Ohm's Law.
    \begin{center}
        \begin{circuitikz}
            \draw
            (0,0)to[R](6,0);
        \end{circuitikz}
    \end{center}

    The inverse of the resistence $R$ is called the conductance $G$
    \begin{equation}
        G\equiv\frac{1}{R}=\pm\frac{i}{v}
    \end{equation}
    The S.I. unit for conductance is 1/Ohm, also refered to as "mho" or "siemens" (Si).

    We could not find a generic relation for the power of a source, since $P=vi$ (assuming PSC holds). Since for a independent voltage source, we don't know the current and vise versa. For a resistor however, assume PSC holds,
    \begin{align}
        P=vi&=(Ri)i=Ri^2\\
        &=v\left(\frac{v}{R}\right)=\frac{v^2}{R}
    \end{align}
    If PSC doesn't hold,
    \begin{align}
        P=-vi&=-(-Ri)i=Ri^2\\
        &=-v\left(-\frac{v}{R}\right)=\frac{v^2}{R}
    \end{align}
    Thus, the power of a resistor is independent of whether or not PSC holds. For physical resistors, $R>0$ hence $P>0$ which means power is always absorbed.
    \item Short Circuit: The limiting behavior for a resistor as $R\to0$. Since $v=Ri$, the voltage is zero independent of the current. This is denoted with a solid line:
    \begin{center}
        \begin{circuitikz}
            \draw (0,0)to(6,0);
        \end{circuitikz}
    \end{center}
    Other names include zero ohm path or ideal conductor.

    \textbf{
    In analysis, all parts of a circuit that are connected using ideal conductor can be considered the same point in the circuit.
    }
    \item Open Circuit: The limiting behavior for a resistor as $R\to\infty$. Since $v=Ri$, the current is zero independent of the voltage. 
    \begin{center}
        \begin{circuitikz}
            \draw (0,0)to(2,0)to[open](4,0)to(6,0);
        \end{circuitikz}
    \end{center}
    For example, the air between the ground and transmission line can be treated like a open circuit, since air is not a conductor.
\end{enumerate}
\subsection{Circuit Analysis Definitions}
\begin{definition}
    A \textbf{Node} is a junction of two or more circuit elements.
    \begin{center}
        \begin{circuitikz}
            \draw (0,0)to[V=$v_s$,-*](0,3);
            \draw (0,3)to[R=$R_2$,-*](4,3)to[R=$R_4$,-*](8,3)to[R=$R_5$](8,0)to(0,0);
            \draw (4,3)to[R=$R_3$](4,0);
            \draw (0,3)to[R=$R_1$,-*](4,0);
        \end{circuitikz}
    \end{center}
    The labelled points are nodes: top-left connects $v_s$, $R_1$ and $R_2$; top-center connects $R_2$, $R_3$, and $R_4$; top-right connects $R_4$ and $R_5$; as well as the region at the bottom (that are connected by short circuit as the junction of $v_s$, $R_1$, $R_3$, $R_5$).
\end{definition}
\section{Lecture 5}
\subsection{Circuit Analysis Definitions cont.}
\begin{definition}
    Start moving from one node towards the other nodes. As long as no node is passed (\textbf{unless} it is a loop when start and end nodes are the same) more than once, the set of nodes passed is called a \textbf{Path}. 
    \begin{center}
        \begin{circuitikz}
            \draw (0,0)to[V=$v_s$,-*](0,3)node[left]{$N_1$};
            \draw (0,3)to[R=$R_2$,-*](4,3)to[R=$R_4$,-*](8,3)to[R=$R_5$](8,0)to(0,0);
            \draw (4,3)to[R=$R_3$](4,0);
            \draw (0,3)to[R=$R_1$,-*](4,0);
            \draw[-latex,blue](0,4)--(4,4)--(9,4)--(9,-1)--(4,-1);
            \draw[-latex,red](0,3.6)--(4,3.6)--(8.6,3.6)--(8.6,-0.4)--(4,-0.4)--(4,4);
        \end{circuitikz}
    \end{center}
    The blue one is a path, the red one is not.
\end{definition}
\begin{definition}
    If the beginning and end of a path is one node, that path is a \textbf{loop}.
    \begin{center}
        \begin{circuitikz}
            \draw (0,0)to[V=$v_s$,-*](0,3)node[left]{$N_1$};
            \draw (0,3)to[R=$R_2$,-*](4,3)to[R=$R_4$,-*](8,3)to[R=$R_5$](8,0)to(0,0);
            \draw (4,3)to[R=$R_3$](4,0);
            \draw (0,3)to[R=$R_1$,-*](4,0);
            \draw[-latex,blue](0,4)--(4,4)--(9,4)--(9,-1)--(4,-1)--(0,2);
            \draw[-latex,red](0,3.6)--(4,3.6)--(8.6,3.6)--(8.6,-0.4)--(4.2,-0.4)--(4.2,4);
        \end{circuitikz}
    \end{center}
    The blue one is a loop, the red one is not a loop.
\end{definition}
\begin{example}[1]
    Identify the loops in the circuit above circuit
\end{example}
\begin{sol}
    Arrows are too hard to draw
    \begin{center}
        \begin{circuitikz}
            \draw (0,0)to[V=$v_s$,-*](0,3)node[left]{$N_1$};
            \draw (0,3)to[R=$R_2$,-*](4,3)to[R=$R_4$,-*](8,3)to[R=$R_5$](8,0)to(0,0);
            \draw (4,3)to[R=$R_3$](4,0);
            \draw (0,3)to[R=$R_1$,-*](4,0);
            \draw[-latex,blue](0,3.1)--(4,3.1)--(9,3.1)--(9,-0.1)--(4,-0.1)--(0,2.9);
            \draw[-latex,green](0,3.2)--(4,3.2)--(8.8,3.2)--(8.8,-0.2)--(-0.2,-0.2)--(-0.2,2.9);
        \end{circuitikz}
    \end{center}
\end{sol}
\subsection{Circuit Analysis Laws}
\begin{enumerate}
    \item \textit{Kirchoff's Current Law (KCL)}: The algebraic sum of the currents entering a node is zero.
    \begin{center}
        \begin{circuitikz}
            \draw (0,0)to[short,i=$i_1$,*-](0,1);
            \draw (0.8,0.6)to[short,i=$i_2$](0,0);
            \draw (0.6,-0.8)to[short,i=$i_3$](0,0);
            \draw (0,0)to[short,i=$i_4$](-0.8,0.6);
        \end{circuitikz}
    \end{center}
    Define a sign convention for the algebraic sum. Either positive sign for current for entering the node or leaving the node is sufficient. For instance, define the sign convention as positive current entering the node. Then, KCL requires
    \begin{equation}
        i_1-i_2+i_3-i_4=0\label{ex1:kcl}
    \end{equation}
    An equivilent statement of KCL is: "The sum of the currents entering a node is equal to the sum of the currents leaving the node." This statement would require
    \begin{equation}
        i_2+i_3=i_1+i_4
    \end{equation}
    which is equivilent to \eqref{ex1:kcl}.
    \begin{example}
        Find $I_1$ in the following circuit
        \begin{center}
            \begin{circuitikz}
                \draw (0,-2)to[short,i=$2\text{A}$,-*](1,0)to[short,i=$3\text{A}$](0,2);
            \end{circuitikz}
        \end{center}
    \end{example}
\end{enumerate}



\end{document}