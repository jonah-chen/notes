\documentclass{article}
\usepackage[a4paper, margin=0.5in]{geometry}
\usepackage{amsmath}
\usepackage{amssymb}
\usepackage{physics}
\usepackage[american]{circuitikz}
\usepackage{physoly}
\usepackage{siunitx}

\begin{document}
\tableofcontents
\section{Lecture 1}
\subsection{An Electric Circult}
    An electric circult is an interconnection of circuit elements (incl. Conductors, semi-conductors, non-conductors). 

\subsection{Electrical Variables}
    To help us analyze electric circuits, we define several electrical variables:
    \begin{definition}
        Electric currents: The "movement" or rate of change of electrical charge. 
        \begin{equation}
            i\equiv\frac{dq}{dt}
        \end{equation} 
        
        \begin{enumerate}
            \item S.I. unit: Ampere = Columb per second.
            \item The current has a direction, and its direction is defined as the direction of positive charge. (Can be shown with $\leftarrow$ or $\rightarrow$)
        \end{enumerate}
    \end{definition}
    When performing circuit analysis, the direction of positive current is either given or we are required to guess a direction of positive current and verify it later. A positive current means the direction you guessed is correct. A negative current means the actual direction is the direction opposiing the current.
    \begin{definition}
        Voltage: The energy required for 1C of charge between point A and B in a circuit is called the voltage between points A and B. 
        \begin{equation}
            V\equiv\frac{dw}{dq}
        \end{equation} 
        \begin{enumerate}
            \item S.I. unit: Volt = Joule per columb.
            \item The voltage also have polarity ($+$ or $-$). Positive polarity means energy is consumed when the charge moves from A to B.
        \end{enumerate}
    \end{definition}
    If the polarity is not given, guess an polarity and a positive voltage indicates the guess is correct and a negative voltage indicates the guess is incorrect. 
    \begin{definition}
        Power: The rate of delivering or absorbing energy.
        \begin{equation}
            p\equiv\frac{dw}{dt}
        \end{equation}
        \begin{enumerate}
            \item S.I. unit: Watt = Joule per second.
            \item Using chain rule, 
            \begin{equation}
                p=\frac{dw}{dt}=\frac{dw}{dq}\frac{dq}{dt}=iv
            \end{equation}
        \end{enumerate}
    \end{definition}
\textbf{Note:} When a problem asks for a current or voltage, the direction/polarity \textbf{must} be indicated otherwise full credit will not be given.
\subsection{Passive Sign Convention}
\begin{definition}
    For a pair of v and i, PSC holds if the current direction \textbf{enters} the positive side of voltage polarity. If PSC holds, $p=+vi$; else $p=-vi$ 
    \begin{center}
        \begin{circuitikz}
            \draw (0,0) to[generic,o-o](5,0);
            \draw (7,0) to[generic,o-o](12,0);
            \draw (1.5,0.2) node{$+$};
            \draw (3.5,0.2) node{$-$};
            \draw (8.5,0.2) node{$+$};
            \draw (10.5,0.2) node{$-$};
            \draw[-latex](0.5,0)--(1,0)node[below]{$i_1$};
            \draw[-latex](8.5,0)--(8,0)node[below]{$i_2$};
        \end{circuitikz}
    \end{center}
    PSC holds on the left and does not hold on the right.
\end{definition}
\noindent The consequences of this convention are
\begin{enumerate}
    \item $p>0\implies$power is absorbed
    \item $p<0\implies$power is delivered (generated by the circuit element)
\end{enumerate}        

\section{Lecture 2}
\begin{example}[1]
    Identify the devices that generate power in the circuit below and find the unknowns in the table.
    \begin{center}
        \begin{circuitikz}
            \draw 
            (0,0) to[generic,V=$v_1$] (0,4) 
            to[generic,V=$v_3$] (6,4);
            \draw
            (0,0) to (6,0)
            to[generic,V=$v_4$] (6,4);
            \draw
            (6,0) to (12,0) 
            to[generic,V=$v_5$] (12,4)
            to (6,4);
            \draw
            (0,4) to[generic,V=$v_2$] (6,0);
        \end{circuitikz}
    \end{center}
\end{example}
\begin{theorem}
    \textbf{Conservation of Power} states that the algebraic sum of the power of all elements in a circuit is zero. (algebraic means the signs are preserved)
    \begin{equation}
        \sum p=0
    \end{equation}
    \begin{prooof}
        Will not be presented at this point in the course.
    \end{prooof}
\end{theorem}

\begin{example}[2]
    \begin{center}
        \begin{circuitikz}
            \draw
            (0,0)node[anchor=east]{Cir.1} to[generic,o-o] (8,0)node[anchor=west]{Cir.2};
            \draw[-latex] (1,0)--(2,0) node[below]{i(t)};
        \end{circuitikz}
    \end{center}
    Given the above circuit and
    \begin{align}
        v(t)&=50(1-e^{-5000t})\text{V}\\
        i(t)&=10e^{-5000t}\text{A}
    \end{align}
    Find the total energy transfered to this device after $t=0$.
\end{example}
\begin{sol}[2]
    PSC holds. Thus,
    \begin{align}
        p(t)&=v(t)i(t)\\
        p(t)&=50(1-e^{-5000t})10(e^{-5000t})\text{W}
    \end{align}
    From the definition of power,
    \begin{align}
        p(t)&=\frac{dw}{dt}\therefore dw=p(t)dt\\
        w&=\int_0^\infty p(t)\dd t=\int_0^\infty 50(1-e^{-5000t})10(e^{-5000t}) \dd t=\frac{1}{20}\text{J}=50\text{mJ}
    \end{align}
\end{sol}
\section{Lecture 3}
\subsection{Circuit Elements}
\begin{enumerate}
    \item Independent sources:
    \begin{enumerate}
        \item \textit{Independent voltage sources:} A circuit element that has a specific voltage independent of the current that flows through it.
        
        For example, the voltage can be a fixed voltage like $v_s=2$V or a variable voltage source like $v_s=2\sin(50t+2)$.

        On diagrams, generic voltage source is shown on the left and fixed (DC) voltage sources is shown on the right. For DC sources, the uppercase letter $V$ is commonly used.
        \begin{center}
            \begin{circuitikz}
                \draw
                (0,0)to[generic,V=,o-o](4,0);
                \draw
                (6,0)node[below]{$+$} to[battery,o-o](10,0)node[below]{$-$};    
            \end{circuitikz}
        \end{center}
        For DC source, the longer lines denotes positive voltage polarity and shorter lines denote negative voltage polarity.

        An sinsudal voltage is drawn as
        \begin{center}
            \begin{circuitikz}
                \draw
                (0,0)to[sinusoidal voltage source,V=,o-o](6,0);
            \end{circuitikz}
        \end{center}

        Questions
        \begin{enumerate}
            \item What is the direction of $i$ for the following voltage source?
            \begin{center}
                \begin{circuitikz}
                    \draw
                    (0,0)to[generic,V=2V,o-o](6,0);
                \end{circuitikz}
            \end{center}
            Answer: You cannot determine.
            \item Does a voltage source always generate power?
            
            Answer: No. Take the voltage source above. Take a current $i_1=2\text{A}\rightarrow$. Then, PSC holds and $p=vi=(2\text{V})(2\text{A})=4\text{W}$. Since $p>0$, the voltage source absorbs power.
        \end{enumerate}

        \item \textit{Independent current source:} It gives a specific current independent of the voltage across it. Can be constant like $i_s=5$A or time dependent like $i_s=10\cos(15t)$
        
        The independent current source is drawn as
        \begin{center}
            \begin{circuitikz}
                \draw
                (0,0)to[I](6,0);
            \end{circuitikz}
        \end{center}

        Questions
        \begin{enumerate}
            \item What is the polarity of voltage across a current source?
            \item Does a current source always generate power?
        \end{enumerate}
        Do not let the word \textit{source} decieve you. A source is not associated with generating power.

        \item Dependent Sources:
        \begin{enumerate}
            \item \textit{Voltage-dependent voltage source:}
            \begin{center}
                \begin{circuitikz}
                    \draw
                    (0,0)to[cV,cV=$Kv_x$,o-o](6,0);
                \end{circuitikz}
            \end{center}
            The voltage of this source depends on voltages somewhere else in the circuit. i.e. $v_s=Kv_s$
            
            \item \textit{Current-dependent voltage source:}
            \begin{center}
                \begin{circuitikz}
                    \draw
                    (0,0)to[cI,cI=$Ki_x$,o-o](6,0);
                \end{circuitikz}
            \end{center}
            The voltage of this source depends on the current somewhere else in a circuit i.e. $v_s=k i_x$. Note k has the dimensions Current/Voltage.
            \item \textit{Voltage-dependent current source:}
            \item \textit{Current-dependent current source:}
        \end{enumerate}
    \end{enumerate}
\end{enumerate}
\section{Lecture 4}
\subsection{Circuit Elements (cont.)}
\begin{enumerate}
    \item Resistor: A circuit element that has keeps a constant ratio between a voltage and current, such that
    \begin{equation}
        R\equiv\frac{v}{i}
    \end{equation}
    S.I. unit: Volt per Amp or Ohm ($\Omega$).

    The resistor is drawn like this. If PSC holds, $v=Ri$ else $v=-Ri$. This is also known as Ohm's Law.
    \begin{center}
        \begin{circuitikz}
            \draw
            (0,0)to[R](6,0);
        \end{circuitikz}
    \end{center}

    The inverse of the resistence $R$ is called the conductance $G$
    \begin{equation}
        G\equiv\frac{1}{R}=\pm\frac{i}{v}
    \end{equation}
    The S.I. unit for conductance is 1/Ohm, also refered to as "mho" or "siemens" (Si).

    We could not find a generic relation for the power of a source, since $P=vi$ (assuming PSC holds). Since for a independent voltage source, we don't know the current and vise versa. For a resistor however, assume PSC holds,
    \begin{align}
        P=vi&=(Ri)i=Ri^2\\
        &=v\left(\frac{v}{R}\right)=\frac{v^2}{R}
    \end{align}
    If PSC doesn't hold,
    \begin{align}
        P=-vi&=-(-Ri)i=Ri^2\\
        &=-v\left(-\frac{v}{R}\right)=\frac{v^2}{R}
    \end{align}
    Thus, the power of a resistor is independent of whether or not PSC holds. For physical resistors, $R>0$ hence $P>0$ which means power is always absorbed.
    \item Short Circuit: The limiting behavior for a resistor as $R\to0$. Since $v=Ri$, the voltage is zero independent of the current. This is denoted with a solid line:
    \begin{center}
        \begin{circuitikz}
            \draw (0,0)to(6,0);
        \end{circuitikz}
    \end{center}
    Other names include zero ohm path or ideal conductor.

    \textbf{
    In analysis, all parts of a circuit that are connected using ideal conductor can be considered the same point in the circuit.
    }
    \item Open Circuit: The limiting behavior for a resistor as $R\to\infty$. Since $v=Ri$, the current is zero independent of the voltage. 
    \begin{center}
        \begin{circuitikz}
            \draw (0,0)to(2,0)to[open](4,0)to(6,0);
        \end{circuitikz}
    \end{center}
    For example, the air between the ground and transmission line can be treated like a open circuit, since air is not a conductor.
\end{enumerate}
\subsection{Circuit Analysis Definitions}
\begin{definition}
    A \textbf{Node} is a junction of two or more circuit elements.
    \begin{center}
        \begin{circuitikz}
            \draw (0,0)to[V=$v_s$,-*](0,3);
            \draw (0,3)to[R=$R_2$,-*](4,3)to[R=$R_4$,-*](8,3)to[R=$R_5$](8,0)to(0,0);
            \draw (4,3)to[R=$R_3$](4,0);
            \draw (0,3)to[R=$R_1$,-*](4,0);
        \end{circuitikz}
    \end{center}
    The labelled points are nodes: top-left connects $v_s$, $R_1$ and $R_2$; top-center connects $R_2$, $R_3$, and $R_4$; top-right connects $R_4$ and $R_5$; as well as the region at the bottom (that are connected by short circuit as the junction of $v_s$, $R_1$, $R_3$, $R_5$).
\end{definition}
\section{Lecture 5}
\subsection{Circuit Analysis Definitions cont.}
\begin{definition}
    Start moving from one node towards the other nodes. As long as no node is passed (\textbf{unless} it is a loop when start and end nodes are the same) more than once, the set of nodes passed is called a \textbf{Path}. 
    \begin{center}
        \begin{circuitikz}
            \draw (0,0)to[V=$v_s$,-*](0,3)node[left]{$N_1$};
            \draw (0,3)to[R=$R_2$,-*](4,3)to[R=$R_4$,-*](8,3)to[R=$R_5$](8,0)to(0,0);
            \draw (4,3)to[R=$R_3$](4,0);
            \draw (0,3)to[R=$R_1$,-*](4,0);
            \draw[-latex,blue](0,4)--(4,4)--(9,4)--(9,-1)--(4,-1);
            \draw[-latex,red](0,3.6)--(4,3.6)--(8.6,3.6)--(8.6,-0.4)--(4,-0.4)--(4,4);
        \end{circuitikz}
    \end{center}
    The blue one is a path, the red one is not.
\end{definition}
\begin{definition}
    If the beginning and end of a path is one node, that path is a \textbf{loop}.
    \begin{center}
        \begin{circuitikz}
            \draw (0,0)to[V=$v_s$,-*](0,3)node[left]{$N_1$};
            \draw (0,3)to[R=$R_2$,-*](4,3)to[R=$R_4$,-*](8,3)to[R=$R_5$](8,0)to(0,0);
            \draw (4,3)to[R=$R_3$](4,0);
            \draw (0,3)to[R=$R_1$,-*](4,0);
            \draw[-latex,blue](0,4)--(4,4)--(9,4)--(9,-1)--(4,-1)--(0,2);
            \draw[-latex,red](0,3.6)--(4,3.6)--(8.6,3.6)--(8.6,-0.4)--(4.2,-0.4)--(4.2,4);
        \end{circuitikz}
    \end{center}
    The blue one is a loop, the red one is not a loop.
\end{definition}
\begin{example}[1]
    Identify the loops in the circuit above circuit
\end{example}
\begin{sol}[1]
    Arrows are too hard to draw
    \begin{center}
        \begin{circuitikz}
            \draw (0,0)to[V=$v_s$,-*](0,3)node[left]{$N_1$};
            \draw (0,3)to[R=$R_2$,-*](4,3)to[R=$R_4$,-*](8,3)to[R=$R_5$](8,0)to(0,0);
            \draw (4,3)to[R=$R_3$](4,0);
            \draw (0,3)to[R=$R_1$,-*](4,0);
            \draw[-latex,blue](0,3.1)--(4,3.1)--(9,3.1)--(9,-0.1)--(4,-0.1)--(0,2.9);
            \draw[-latex,green](0,3.2)--(4,3.2)--(8.8,3.2)--(8.8,-0.2)--(-0.2,-0.2)--(-0.2,2.9);
        \end{circuitikz}
    \end{center}
\end{sol}
\subsection{Circuit Analysis Laws}
\begin{enumerate}
    \item \textit{Kirchoff's Current Law (KCL)}: The algebraic sum of the currents entering a node is zero.
    \begin{center}
        \begin{circuitikz}
            \draw (0,0)to[short,i=$i_1$,*-](0,1);
            \draw (0.8,0.6)to[short,i=$i_2$](0,0);
            \draw (0.6,-0.8)to[short,i=$i_3$](0,0);
            \draw (0,0)to[short,i=$i_4$](-0.8,0.6);
        \end{circuitikz}
    \end{center}
    Define a sign convention for the algebraic sum. Either positive sign for current for entering the node or leaving the node is sufficient. For instance, define the sign convention as positive current entering the node. Then, KCL requires
    \begin{equation}
        i_1-i_2+i_3-i_4=0\label{ex1:kcl}
    \end{equation}
    An equivilent statement of KCL is: "The sum of the currents entering a node is equal to the sum of the currents leaving the node." This statement would require
    \begin{equation}
        i_2+i_3=i_1+i_4
    \end{equation}
    which is equivilent to \eqref{ex1:kcl}.
    \begin{example}[2]
        Find $I_1$ in the following circuit
        \begin{center}
            \begin{circuitikz}
                \draw (0,-2)to[short,i=$2\text{A}$,-*](1,0)to[short,i=$3\text{A}$](0,2);
                \draw (3,-2)to[short,i=\SI{1}{A}](2,0)to(1,0);
                \draw (2,0)to(4,0)to[short,i=$I_1$](5,2);
                \draw (4,0)to(5,0)to[short,i=\SI{1}{A}](6,2);
                \draw (6,-2)to[short,i=\SI{4}{A}](5,0);
            \end{circuitikz}
        \end{center}
    \end{example}
\end{enumerate}
\section{Lecture 6}
\subsection{Circuit Analysis Laws (cont.)}
\begin{enumerate}
    \item \textit{Kirchoff's Voltage Law (KVL)}: The algebraic sum of the voltages around any loop is zero.
    \begin{enumerate}
        \item If the loop direction enters a voltage source from positive voltage polarity, voltage used for KVL is positive.
        \item Else, the voltage used for KVL is negative.
    \end{enumerate}
    \begin{center}
        \begin{circuitikz}
            \draw (0,4)to[V=$v_1$](0,0)to(6,0)to[V=$v_3$](6,4);
            \draw (0,4)to[V=$v_2$](6,4);
        \end{circuitikz}
    \end{center}
    For the same loop, the starting point and direction of performing the analysis is irrelevent. i.e. starting from the top left in the clockwise direction, KVL requires
    \begin{equation}
        +v_2-v_3-v_1=0
    \end{equation}
    Starting from the top right in the anti-clockwise direction,
    \begin{equation}
        -v_2+v_1+v_3=0
    \end{equation}
    When writing KVL for circuits with resistors, it can be combined with Ohm's law in the following fashion. Note that this is independent of voltage polarity defined for the resistor.
    \begin{enumerate}
        \item When the current direction entering a resistor is the same as the loop direction, the voltage of the resistor for KVL is $+iR$
        \item When the current direction entering a resistor is opposing the loop direction, the voltage of the resistor for KVL is $-iR$ 
    \end{enumerate}
    \begin{example}[1]
        Use KVL and Ohm's law to find the current of the resistor.
        \begin{center}
            \begin{circuitikz}
                \draw (0,3)to[V=\SI{10}{V}](0,0)to(3,0);
                \draw (0,3)to[short,i=$i_1$](3,3)to[R=\SI{2}{\ohm},v=$v_1$](3,0);
            \end{circuitikz}
        \end{center}
    \end{example}
    \begin{sol}[1]
        Applying KVL in the clockwise direction,
        \begin{equation}
            v_1-10=0\implies v_1=\SI{10}{V}
        \end{equation}
        Using Ohm's law. Since current enters the + polarity, PSC holds. Thus,
        \begin{equation}
            v_1=2i_1\implies 10=2i_1\implies i_1=\SI{5}{A}
        \end{equation}
        Using the "shortcut", again applying KVL in a clockwise direction. Since the current direction is the same as the loop direction,
        \begin{equation}
            +2i_1-10=0\implies i_1=\SI{5}{A}
        \end{equation}
    \end{sol}
\end{enumerate}
\subsection{Circuit Analysis Tips}
If a circuit \textbf{does not} include a voltage-dependent current source or a current-dependent voltage source, you may use any of the following units systems to perform your analysis. 
\begin{enumerate}
    \item V, A, $\Omega$, W
    \item V, mA, k$\Omega$, mW
    \item kV, mA, $\Omega$, W
    \item Other consistent units can be used but are not as common.
\end{enumerate}
It is simpler to use a consistent system of units to analyze the circuits. Thus, units consistent with the voltage-dependent current source or a current-dependent voltage source should be used.
\subsection{Some Examples}
\begin{example}[2]
    Find the power of the voltage source.
    \begin{center}
        \begin{circuitikz}
            \draw (0,4)node[left]{N}
            to[V=\SI{30}{V},i=$i_s$,*-](0,0)
            to(6,0)
            to[R=\SI{50}{\ohm}](6,4)
            to[R=\SI{100}{\ohm},i=$i_2$](0,4)
            to[R=\SI{300}{\ohm},i=$i_1$](6,0);
        \end{circuitikz}
    \end{center}
\end{example}
\begin{sol}[2]
    Since PSC does not hold for the voltage source,
    \begin{equation}
        P=-30i_s
    \end{equation}
    KVL for the bottom left loop (clockwise)
    \begin{equation}
        -30+300i_1=0\implies i_1=\SI{0.1}{A}
    \end{equation}
    KVL for the outer loop (clockwise)
    \begin{equation}
        -30-100i_2-50i_2=0\implies i_2=\SI{-0.2}{A}
    \end{equation}
    KCL at node N
    \begin{equation}
        i_s+i_2=i_1\implies i_s=i_1-i_2=0.1-(-0.2)=\SI{0.3}{A}
    \end{equation}
    Find the power
    \begin{equation}
        P=-30i_s=-30(0.3)=\SI{-9}{W}
    \end{equation}
    Since $P<0$, power is generated.
\end{sol}
\begin{example}[3]
    Find the power of each source.
    \begin{center}
        \begin{circuitikz}
            \draw (0,4)
            to[V=\SI{100}{V}](0,0)
            to(8,0)
            to[I=\SI{1}{mA},v=$v_y$](8,4)
            to[R=\SI{22}{\kilo\ohm}](4,4);
            \draw (0,4)
            to[short,i=$i_x$,-*](4,4)node[above]{N}
            to[R=\SI{33}{k\ohm},i=$i_z$](4,0);
        \end{circuitikz}
    \end{center}
\end{example}
\begin{sol}[3]
    KVL for the left loop (clockwise)
    \begin{equation}
        -100+33i_z=0\implies i_z=\SI{3}{mA}
    \end{equation}
    KCL for N, $i_x$ and \SI{1}{mA} are entering N and $i_z=\SI{3}{mA}$ is leaving N.
    \begin{equation}
        i_x+1=i_z\implies i_x=\SI{2}{mA}
    \end{equation}
    For the voltage source, PSC does not hold. Thus,
    \begin{equation}
        P=-vi=-100\times 2=\SI{-200}{mW}
    \end{equation}
    KVL for outer loop (clockwise)
    \begin{equation}
        -100-(1\times 22)+v_y=0\implies v_y=\SI{122}{V}
    \end{equation}
    For the current source, PSC does not hold. Thus,
    \begin{equation}
        P=-vi=-122\times 1=\SI{-122}{mW}
    \end{equation}
\end{sol}
\begin{example}[4]
    Find the power of each source.
    \begin{center}
        \begin{circuitikz}
            \draw (0,0)
            to[I=\SI{2}{A},v=$v_x$](0,3)
            to[R=\SI{100}{\ohm},i=$i_1$,-*](4,3)node[above]{$N_3$}
            to[R=\SI{100}{\ohm},i=$i_2$](8,3)
            to[I=\SI{5}{A}](8,0)
            to(0,0);
            \draw (4,0)to[R=\SI{50}{\ohm},i=$i_3$,*-](4,3);
            \draw (0,3)node[left]{$N_1$}
            to(0,5)
            to[I=\SI{3}{A}](8,5)
            to[short,-*](8,3)node[right]{$N_2$};
        \end{circuitikz}
    \end{center}
\end{example}
\begin{sol}[4]
    KCL at $N_1$
    \begin{equation}
        2=3+i_1\implies i_1=\SI{-1}{A}
    \end{equation}
    KCL at $N_2$
    \begin{equation}
        3+i_2=5\implies i_2=\SI{2}{A}
    \end{equation}
    KCL at $N_2$
    \begin{equation}
        i_1+i_3=i_2\implies i_3=\SI{3}{A}
    \end{equation}
    KVL for the bottom left loop (clockwise)
    \begin{equation}
        -v_x+100i_1-50i_3=0\implies v_x=100\times -1-50\times 3=\SI{-250}{V}
    \end{equation}
    The rest is left as an exercise.
\end{sol}
\begin{example}[5]
    \begin{center}
        \begin{circuitikz}
            \draw (3,0)
            to(0,0)
            to[I=\SI{2}{mA}](0,4)
            to (3,4)
            to[R=\SI{4.7}{k\ohm},v=$v_x$](3,0)
            to[short,i=$i_2$] (7,0)
            to(10,0);
            \draw (10,4)to[R=\SI{3}{k\ohm},i=$i_1$](10,0);
        \end{circuitikz}
    \end{center}
\end{example}
\begin{example}[6]
    \begin{center}
        \begin{circuitikz}
            \draw(0,3)
            to[V=\SI{2}{V}](0,0)
            to (2,0);
            \draw (2,3)
            to[R=\SI{6}{\ohm},v=$v_2$,i=$i_2$](2,0)
            to(9,0)
            to[R=\SI{5}{\ohm},v=$v_1$,i=$i_1$](9,3)
            to(7,3);
            \draw (0,3)
            to(2,3)
            to[cI=$5v_2$](7,3)
            to[cV=$5i_2$,i=$i_x$](7,0);
        \end{circuitikz}
    \end{center}
\end{example}
\begin{sol}[6]
    Use KVL on the left most loop (clockwise).
    \begin{equation}
        -2+6i_2=0\implies i_2=\SI{0.33}{A}
    \end{equation}
    Using Ohm's law on the resistor, PSC holds thus
    \begin{equation}
        v_2=6i_2=\SI{2}{V}
    \end{equation}
    The current provided by the dependent current source would be \SI{10}{A} and the voltage provided by the dependent voltage source would be \SI{1.67}{V}. 

    Using KVL on the right most loop (anti-clockwise),
    \begin{equation}
        v_1+5i_2=0\implies v_1=-5i_2=\SI{-1.67}{V}
    \end{equation}
    Using Ohm's law on this resistor to find $i_1$. As PSC holds,
    \begin{equation}
        i_1=v_1/5=\SI{0.33}{A}
    \end{equation}
    Finally using KCL at the junction,
    \begin{equation}
        i_x+i_1+5v_2=0\implies i_x=\SI{-10.33}{A}
    \end{equation}    
\end{sol}
\section{Lecture 7}
\subsection{Equivilent circuits for parallel and series resistors}
\begin{center}
    \begin{circuitikz}
        \draw (0,0)
        to[short,o-](4,0)
        to[R=$R_4$](4,3)
        to[R=$R_1$](1,3)
        to[short,-o](0,3);
    \end{circuitikz}
\end{center}
\begin{definition}
    Series Connection: Two circuit elements are connect in series if and only if they are connected back to back, and at their point of connection, there is no other current path.
    \begin{center}
        \begin{circuitikz}
            \draw (0,0)
            to[R=$R_1$](2,0)
            to[R=$R_2$](4,0);
            \draw (7,0)
            to[R=$R_3$](9,0)
            to[R=$R_4$](11,0);
            \draw (9,0)
            to(9,1);
        \end{circuitikz}
    \end{center}
    $R_1$ and $R_2$ are connected in series. $R_3$ and $R_4$ are also connected in series since the path between the two resistence is a open circuit, thus the current is zero so it is not a current path.
\end{definition}
\begin{theorem}
    For resistors in series, the equivilent resistence is 
    \begin{equation}
        R_{eq}=\sum_kR_k
    \end{equation}
\end{theorem}
\begin{prooof}
    Take a circuit with 2 resistors in series.
    
    Using KVL (clockwise),
    \begin{equation}
        -v_{tot}+v_1+v_2=0
    \end{equation}
    Using Ohm's law, 
    \begin{align}
        v_1&=R_1i_{tot}\\
        v_2&=R_2i_{tot}
    \end{align} 
    Combining the results,
    \begin{equation}
        v_{tot}=v_1+v_2=(R_1+R_2)i_{tot}
    \end{equation}
    For an equvilent resistor from Ohm's law,
    \begin{equation}
        v_{tot}=R_{eq}i_{tot}\implies R_{eq}=R_1+R_2
    \end{equation}
    From induction, the equivilent resistence of any series connected resistors is equal to the sum of their resistences.
\end{prooof}
\begin{definition}
    Parallel connection: Two circuit elements are connected in parallel if they share two common nodes.
    \begin{center}
        \begin{circuitikz}
            \draw (0,0)
            to[short,o-](4,0)
            to[R=$R_2$](4,2)
            to[short,-o](0,2);
            \draw (2,0)
            to[R=$R_1$](2,2);
        \end{circuitikz}
    \end{center}
\end{definition}
\begin{theorem}
    For parellel resistors, the equivilent resistence is 
    \begin{equation}
        R_{eq}=\left[\sum_k\frac{1}{R_k}\right]^{-1}
    \end{equation}
\end{theorem}
\begin{prooof}
    \begin{center}
        \begin{circuitikz}
            \draw (0,0)
            to[short,o-](4,0)
            to[R=$R_2$](4,2)
            to[short,-o](0,2);
            \draw (2,0)
            to[R=$R_1$](2,2);
        \end{circuitikz}
    \end{center}
    Write KCL at N
    \begin{equation}
        i_{tot}=i_1+i_2
    \end{equation}
    Using KVL
    \begin{equation}
        -v_{tot}+v_1=0\implies v_{tot}=v_1=v_2
    \end{equation}
    Using Ohm's law, PSC holds
    \begin{equation}
        i_{tot}=\frac{v_1}{R_1}+\frac{v_2}{R_2}=\left(\frac{1}{R_1}+\frac{1}{R_2}\right)v_{tot}
    \end{equation}
    Using ohm's law for the equivilent resistor,
    \begin{equation}
        i_{tot}=\frac{v_{tot}}{R_{eq}}\implies \frac{1}{R_{eq}}=\frac{1}{R_1}+\frac{1}{R_2}\implies R_{eq}=\frac{R_1R_2}{R_1+R_2}
    \end{equation}
\end{prooof}

Consider two resistors in parellel with $R_2=0$. Thus, 
\begin{equation}
    R_{eq}=\frac{R_1R_2}{R_1+R_2}=\frac{0}{R_1}=0
\end{equation}
You can consider that the current is "sane". When given an option to flow with no resistence, it will take that path. Thus, if $R_2=0$, no current will flow through $R_1$ and the equivilent resistence is 0

Correlary: Recall the definition of conductance is $G=1/R$. Thus,
\begin{equation}
    G_{eq}=\sum_kG_k
\end{equation}
\begin{example}[1]
    Find the equivilent resistence between A and B
    \begin{center}
        \begin{circuitikz}
            \draw (0,0)node[left]{B}
            to[short,o-](7,0)
            to[R=\SI{15}{k\ohm}](7,2)
            to(5,2)
            to[R=\SI{47}{k\ohm}](2,2)
            to[short,-o](0,2)node[left]{A};
            \draw (2,0)
            to[R=\SI{56}{k\ohm}](2,2);
            \draw (5,0)
            to[R=\SI{15}{k\ohm}](5,2);
        \end{circuitikz}
    \end{center}
\end{example}
\begin{sol}[1]
    Simplify the circuit in several steps (starting from the right)
    \begin{enumerate}
        \item The two right most resistors are correct in parellel.
        \begin{center}
            \begin{circuitikz}
                \draw (0,0)node[left]{B}
                to[short,o-](5,0)
                to[R=$15\|15$ \SI{7.5}{k\ohm}](5,2)
                to[R=\SI{47}{k\ohm}](2,2)
                to[short,-o](0,2)node[left]{A};
                \draw (2,0)
                to[R=\SI{56}{k\ohm}](2,2);
            \end{circuitikz}
        \end{center}
        \item 
        \begin{center}
            \begin{circuitikz}
                \draw (0,0)node[left]{B}
                to[short,o-](5,0)
                to(5,2)
                to[R=\SI{54.5}{k\ohm}](2,2)
                to[short,-o](0,2)node[left]{A};
                \draw (2,0)
                to[R=\SI{56}{k\ohm}](2,2);
            \end{circuitikz}
        \end{center}
        \item \begin{equation}
            R_{eq}=56\|54.5=\frac{56\times 54.5}{56+54.5}=\SI{27.62}{k\ohm}
        \end{equation}
    \end{enumerate}
    Note that in the original circuit, the \SI{15}{k\ohm} resistors are \textbf{not} connected in series to the \SI{47}{k\ohm} resistor as there exist (one) current path between them.
\end{sol}
\begin{example}[2]
    Select $R$ such that $R_{AB}=R_L$
    \begin{center}
        \begin{circuitikz}
            \draw (-1,0)node[left]{B}
            to[short,o-](5,0)
            to[R=$R_L$](5,2)
            to[R=$R$](2,2)
            to[R=$R$,-o](-1,2)node[left]{A};
            \draw (2,0)
            to[R=$4R$](2,2);
        \end{circuitikz}
    \end{center}
\end{example}
\section{Lecture 8}
\subsection{Voltage Division}
Suppose there are two resistor connected in series. With voltage through both as $v_T$ and $i_T$.
\begin{center}
    \begin{circuitikz}
        \draw (0,0)node[above]{$-$}
        to(2,0);
        \draw (0,4)node[below]{$+$}
        to[short,i=$i_T$] (2,4)
        to[R=$R_1$,v=$v_1$](2,2)
        to[R=$R_2$,v=$v_2$](2,0);
        \draw (0,2)node{$v_T$};
    \end{circuitikz}
\end{center}
Then the voltage division principle states 
\begin{equation}
    v_1=\frac{R_1}{R_1+R_2}v_T
\end{equation}
In general for $N$ series connected resistors,
\begin{equation}
    v_i=\frac{R_i}{\sum_kR_k}v_T
\end{equation}
Note the polarity of the voltage $v_i$ \textbf{must} be the same as $v_T$, otherwise the $-$ sign must be introduced.

As a consequence, the resistor with the higher resistence will have the higher voltage. 

\begin{prooof}
    The equivilent resistence of the two resistors is $R_T=R_1+R_2$. Using Ohm's law, $v_T=i_T(R_1+R_2)$. 
\end{prooof}

\subsection{Current Division}
\begin{derivation}
Suppose there are two resistors connected in parellel.
\begin{center}
    \begin{circuitikz}
        \draw (0,0)
        to[short,o-,i=$i_T$](2,0)
        to(4,0)
        to[R=$R_2$,i=$i_2$,v=$v_T$](4,2)
        to[short,-o](0,2);
        \draw (2,0)
        to[R=$R_1$,i=$i_1$,v=$v_T$](2,2);
    \end{circuitikz}
\end{center}
    Use Ohm's Law
    \begin{equation}
        v_T=R_1i_1
    \end{equation}
    \begin{equation}
        v_T=\frac{R_1R_2}{R_1+R_2}
    \end{equation}
    Since the LHS is the same, the RHS must be the same as well. Thus,
    \begin{equation}
        i_1=\frac{R_2}{R_1+R_2}i_T\label{current:div}
    \end{equation}
    Equation \eqref{current:div} is known as the current division principle. This can also be expressed in terms of conductances
    \begin{equation}
        i_1=\frac{G_1}{G_1+G_2}i_T
    \end{equation}
    For the general case of $N$ parellel resistors, it is possible to use the equation with the conductances similar to the voltage division principle.
    \begin{equation}
        i_i=\frac{G_i}{\sum_kG_k}i_T
    \end{equation}
    It is possible to write the other resistors as an equivilent resistor with resistence $R_{eq}'$, then the current at the resistor of interest $R$ is
    \begin{equation}
        i=\frac{R_{eq}'}{R+R_{eq}'}i_T
    \end{equation}
\end{derivation}

The directions of $i_T$ and $i_1$ \textbf{must} be similar. Otherwise, the $-$ sign must be introduced. 
\begin{example}
    Find $v_0$ and $i_0$
    \begin{center}
        \begin{circuitikz}
            \draw (0,0)
            to(4,0)
            to[R=$R_2$,i=$i_2$,v=$v_T$](4,2)
            to[short,-o](0,2);
            \draw (2,0)
            to[R=$R_1$,i=$i_1$,v=$v_T$](2,2);
        \end{circuitikz}
    \end{center}
\end{example}
\section{Lecture}
\subsection{Nodal Analysis}
Objective: Find node voltages.

Methodology: KCL for all the nod in terms of the \textbf{node voltage}: The voltage with respect to a reference node or ground node.
\begin{definition}
    \begin{enumerate}
        \item $V_{AB}$ is the voltage with the + polarity at point A and $-$ polarity at point B. 
        \item $V_A$ is the voltage with the $+$ polarity at point A and $-$ at the reference node or ground.
        \item By definition, the voltage of the ground node is equal to zero.
    \end{enumerate}
\end{definition}

\begin{example}[1]
    Perform nodal analysis on the following circuit.
    \begin{center}
        \begin{circuitikz}
            \draw (0,0)
            to[I=\SI{1}{A}](0,3)
            to[short,-*](2,3)node[above]{$v_1$}
            to[R=\SI{6}{\ohm},-*,i=$i_y$](8,3)node[above]{$v_2$}
            to(10,3)
            to[I=\SI{3}{A}](10,0)
            to(0,0);
            \draw (2,3)to[R=\SI{4}{\ohm},i=$i_x$](2,0);
            \draw (8,0)to[R=\SI{2}{\ohm}](8,3);
        \end{circuitikz}
    \end{center}
\end{example}

\begin{proposition}
    The common convention is to assume a negative sign for current entering node and positive sign for current leaving node. It is also customary to write the current leaving the node at a resistor.
\end{proposition}
\begin{sol}[1]
    Using ohm's law, $i_x=v_1/4$, $i_y=(v_1-v_2)/6$. (labelled for analysis of node 1)

    Execute KCL for node 1
    \begin{equation}
        -1+\frac{v_1}{4}+\frac{v_1-v_2}{6}=0
    \end{equation}
    Execute KCL for node 2. It is recommended to start fresh for each node. 
    \begin{equation}
        +3+\frac{v_2}{2}+\frac{v_2-v_1}{6}=0
    \end{equation}
    Now there are two equations for two unknowns. This can be solved using computer to obtain
    $v_1=\SI{2/3}{V},v_2=\SI{-13/2}{V}$. Note that $v_1$ represents the voltage measured by using a voltmeter with the positive terminal of node 1 and negative terminal at the ground. 
    
    The power of nodal analysis is that now it is possible to easily find the voltage between any two points in the circuit. For example, the voltage of the \SI{3}{A} is just $v_2-0=\SI{-13/2}{A}$
\end{sol}

\begin{example}[2]
    \begin{center}
        \begin{circuitikz}
            \draw (0,3)
            to[V=\SI{10}{V},*-] (0,0)
            to(3,0)
            to[R=\SI{4}{\ohm}](3,3)
            to[R=\SI{1}{\ohm}](0,3)node[above]{$v_1$};
            \draw (3,3)node[above]{$v_2$}
            to[R=\SI{8}{\ohm},*-*,i=$i_x$](6,3)node[above]{$v_3$}
            to[R=\SI{2}{\ohm}](6,0);
            \draw (6,3)
            to[R=\SI{5}{\ohm},-*](9,3)node[above]{$v_4$}
            to[V=\SI{20}{V}](9,0);
            \draw (3,0)to(9,0);
        \end{circuitikz}
    \end{center}
\end{example}
\begin{sol}
    In this circuit, there are voltage sources! Hence, we do not know the current of the at certain nodes thus writing KCL writing there would not be productive. However, the voltage at nodes 1 and 4 are actually known!
    \begin{align}
        v_1&=+\SI{10}{V}\\
        v_2&=+\SI{20}{V}
    \end{align}
    \textbf{Note} to pay attention to the direction the voltage sources are oriented.

    KCL for node 2
    \begin{equation}
        \frac{v_2-v_1}{1}+\frac{v_2-0}{4}+\frac{v_2-v_3}{8}=0
    \end{equation}
    KCL at node 3
    \begin{equation}
        \frac{v_3-v_4}{5}+\frac{v_3+0}{2}+\frac{v_3-v_2}{8}=0
    \end{equation}
    It may seem like there are too many unknowns, but $v_1$ and $v_2$ are already known. Hence, only $v_2=\SI{7.82}{V},v_3=\SI{6.03}{V}$ needs to be found.

    To find $i_x$, use the ohm's law
    \begin{equation}
        i_x=\frac{v_2-v_3}{8}=\SI{0.223}{A}
    \end{equation}
\end{sol}
\section{Lecture}
\subsection{Nodal Analysis (cont.)---Circuits with Dependent Sources}
\begin{itemize}
    \item Before writing the KCL for dependent source, first write the parameters it depends on in terms of the \textit{node voltages}.
\end{itemize}

\begin{example}
    Perform nodal analysis on the circuit.
    \begin{center}
        \begin{circuitikz}
            \draw(0,3)
            to[V=\SI{4}{V}](0,0)
            to(6,0)
            to[cI=$2v_x$](6,3)
            to(6,5)
            to[R=\SI{2}{\ohm},v=$v_x$](0,5);
            \draw(0,5)
            to[short,-*](0,3)node[left]{$v_A$};
            \draw(3,3)node[above]{$v_B$}
            to[I=\SI{7}{A},*-](0,3);
            \draw (3,3) to[R=\SI{3}{\ohm}] (3,0);
            \draw (3,3)to[R=\SI{1}{\ohm},-*](6,3)node[right]{$v_C$};
        \end{circuitikz}
    \end{center}
\end{example}

\
% mouse hover event
% get mouse position (system.windows.form.cursor.positon)
% Form.location (where the form is on the screen)
% click action (for the form)

% image box
% C# bitmap for updating image

\end{document}